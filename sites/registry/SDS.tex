% Options for packages loaded elsewhere
\PassOptionsToPackage{unicode}{hyperref}
\PassOptionsToPackage{hyphens}{url}
%
\documentclass[
  11pt,
  a4paper,
]{book}

\usepackage{amsmath,amssymb}
\usepackage{iftex}
\ifPDFTeX
  \usepackage[T1]{fontenc}
  \usepackage[utf8]{inputenc}
  \usepackage{textcomp} % provide euro and other symbols
\else % if luatex or xetex
  \usepackage{unicode-math}
  \defaultfontfeatures{Scale=MatchLowercase}
  \defaultfontfeatures[\rmfamily]{Ligatures=TeX,Scale=1}
\fi
\usepackage{lmodern}
\ifPDFTeX\else  
    % xetex/luatex font selection
\fi
% Use upquote if available, for straight quotes in verbatim environments
\IfFileExists{upquote.sty}{\usepackage{upquote}}{}
\IfFileExists{microtype.sty}{% use microtype if available
  \usepackage[]{microtype}
  \UseMicrotypeSet[protrusion]{basicmath} % disable protrusion for tt fonts
}{}
\makeatletter
\@ifundefined{KOMAClassName}{% if non-KOMA class
  \IfFileExists{parskip.sty}{%
    \usepackage{parskip}
  }{% else
    \setlength{\parindent}{0pt}
    \setlength{\parskip}{6pt plus 2pt minus 1pt}}
}{% if KOMA class
  \KOMAoptions{parskip=half}}
\makeatother
\usepackage{xcolor}
\setlength{\emergencystretch}{3em} % prevent overfull lines
\setcounter{secnumdepth}{5}
% Make \paragraph and \subparagraph free-standing
\makeatletter
\ifx\paragraph\undefined\else
  \let\oldparagraph\paragraph
  \renewcommand{\paragraph}{
    \@ifstar
      \xxxParagraphStar
      \xxxParagraphNoStar
  }
  \newcommand{\xxxParagraphStar}[1]{\oldparagraph*{#1}\mbox{}}
  \newcommand{\xxxParagraphNoStar}[1]{\oldparagraph{#1}\mbox{}}
\fi
\ifx\subparagraph\undefined\else
  \let\oldsubparagraph\subparagraph
  \renewcommand{\subparagraph}{
    \@ifstar
      \xxxSubParagraphStar
      \xxxSubParagraphNoStar
  }
  \newcommand{\xxxSubParagraphStar}[1]{\oldsubparagraph*{#1}\mbox{}}
  \newcommand{\xxxSubParagraphNoStar}[1]{\oldsubparagraph{#1}\mbox{}}
\fi
\makeatother

\usepackage{color}
\usepackage{fancyvrb}
\newcommand{\VerbBar}{|}
\newcommand{\VERB}{\Verb[commandchars=\\\{\}]}
\DefineVerbatimEnvironment{Highlighting}{Verbatim}{commandchars=\\\{\}}
% Add ',fontsize=\small' for more characters per line
\usepackage{framed}
\definecolor{shadecolor}{RGB}{241,243,245}
\newenvironment{Shaded}{\begin{snugshade}}{\end{snugshade}}
\newcommand{\AlertTok}[1]{\textcolor[rgb]{0.68,0.00,0.00}{#1}}
\newcommand{\AnnotationTok}[1]{\textcolor[rgb]{0.37,0.37,0.37}{#1}}
\newcommand{\AttributeTok}[1]{\textcolor[rgb]{0.40,0.45,0.13}{#1}}
\newcommand{\BaseNTok}[1]{\textcolor[rgb]{0.68,0.00,0.00}{#1}}
\newcommand{\BuiltInTok}[1]{\textcolor[rgb]{0.00,0.23,0.31}{#1}}
\newcommand{\CharTok}[1]{\textcolor[rgb]{0.13,0.47,0.30}{#1}}
\newcommand{\CommentTok}[1]{\textcolor[rgb]{0.37,0.37,0.37}{#1}}
\newcommand{\CommentVarTok}[1]{\textcolor[rgb]{0.37,0.37,0.37}{\textit{#1}}}
\newcommand{\ConstantTok}[1]{\textcolor[rgb]{0.56,0.35,0.01}{#1}}
\newcommand{\ControlFlowTok}[1]{\textcolor[rgb]{0.00,0.23,0.31}{\textbf{#1}}}
\newcommand{\DataTypeTok}[1]{\textcolor[rgb]{0.68,0.00,0.00}{#1}}
\newcommand{\DecValTok}[1]{\textcolor[rgb]{0.68,0.00,0.00}{#1}}
\newcommand{\DocumentationTok}[1]{\textcolor[rgb]{0.37,0.37,0.37}{\textit{#1}}}
\newcommand{\ErrorTok}[1]{\textcolor[rgb]{0.68,0.00,0.00}{#1}}
\newcommand{\ExtensionTok}[1]{\textcolor[rgb]{0.00,0.23,0.31}{#1}}
\newcommand{\FloatTok}[1]{\textcolor[rgb]{0.68,0.00,0.00}{#1}}
\newcommand{\FunctionTok}[1]{\textcolor[rgb]{0.28,0.35,0.67}{#1}}
\newcommand{\ImportTok}[1]{\textcolor[rgb]{0.00,0.46,0.62}{#1}}
\newcommand{\InformationTok}[1]{\textcolor[rgb]{0.37,0.37,0.37}{#1}}
\newcommand{\KeywordTok}[1]{\textcolor[rgb]{0.00,0.23,0.31}{\textbf{#1}}}
\newcommand{\NormalTok}[1]{\textcolor[rgb]{0.00,0.23,0.31}{#1}}
\newcommand{\OperatorTok}[1]{\textcolor[rgb]{0.37,0.37,0.37}{#1}}
\newcommand{\OtherTok}[1]{\textcolor[rgb]{0.00,0.23,0.31}{#1}}
\newcommand{\PreprocessorTok}[1]{\textcolor[rgb]{0.68,0.00,0.00}{#1}}
\newcommand{\RegionMarkerTok}[1]{\textcolor[rgb]{0.00,0.23,0.31}{#1}}
\newcommand{\SpecialCharTok}[1]{\textcolor[rgb]{0.37,0.37,0.37}{#1}}
\newcommand{\SpecialStringTok}[1]{\textcolor[rgb]{0.13,0.47,0.30}{#1}}
\newcommand{\StringTok}[1]{\textcolor[rgb]{0.13,0.47,0.30}{#1}}
\newcommand{\VariableTok}[1]{\textcolor[rgb]{0.07,0.07,0.07}{#1}}
\newcommand{\VerbatimStringTok}[1]{\textcolor[rgb]{0.13,0.47,0.30}{#1}}
\newcommand{\WarningTok}[1]{\textcolor[rgb]{0.37,0.37,0.37}{\textit{#1}}}

\providecommand{\tightlist}{%
  \setlength{\itemsep}{0pt}\setlength{\parskip}{0pt}}\usepackage{longtable,booktabs,array}
\usepackage{calc} % for calculating minipage widths
% Correct order of tables after \paragraph or \subparagraph
\usepackage{etoolbox}
\makeatletter
\patchcmd\longtable{\par}{\if@noskipsec\mbox{}\fi\par}{}{}
\makeatother
% Allow footnotes in longtable head/foot
\IfFileExists{footnotehyper.sty}{\usepackage{footnotehyper}}{\usepackage{footnote}}
\makesavenoteenv{longtable}
\usepackage{graphicx}
\makeatletter
\def\maxwidth{\ifdim\Gin@nat@width>\linewidth\linewidth\else\Gin@nat@width\fi}
\def\maxheight{\ifdim\Gin@nat@height>\textheight\textheight\else\Gin@nat@height\fi}
\makeatother
% Scale images if necessary, so that they will not overflow the page
% margins by default, and it is still possible to overwrite the defaults
% using explicit options in \includegraphics[width, height, ...]{}
\setkeys{Gin}{width=\maxwidth,height=\maxheight,keepaspectratio}
% Set default figure placement to htbp
\makeatletter
\def\fps@figure{htbp}
\makeatother
% definitions for citeproc citations
\NewDocumentCommand\citeproctext{}{}
\NewDocumentCommand\citeproc{mm}{%
  \begingroup\def\citeproctext{#2}\cite{#1}\endgroup}
\makeatletter
 % allow citations to break across lines
 \let\@cite@ofmt\@firstofone
 % avoid brackets around text for \cite:
 \def\@biblabel#1{}
 \def\@cite#1#2{{#1\if@tempswa , #2\fi}}
\makeatother
\newlength{\cslhangindent}
\setlength{\cslhangindent}{1.5em}
\newlength{\csllabelwidth}
\setlength{\csllabelwidth}{3em}
\newenvironment{CSLReferences}[2] % #1 hanging-indent, #2 entry-spacing
 {\begin{list}{}{%
  \setlength{\itemindent}{0pt}
  \setlength{\leftmargin}{0pt}
  \setlength{\parsep}{0pt}
  % turn on hanging indent if param 1 is 1
  \ifodd #1
   \setlength{\leftmargin}{\cslhangindent}
   \setlength{\itemindent}{-1\cslhangindent}
  \fi
  % set entry spacing
  \setlength{\itemsep}{#2\baselineskip}}}
 {\end{list}}
\usepackage{calc}
\newcommand{\CSLBlock}[1]{\hfill\break\parbox[t]{\linewidth}{\strut\ignorespaces#1\strut}}
\newcommand{\CSLLeftMargin}[1]{\parbox[t]{\csllabelwidth}{\strut#1\strut}}
\newcommand{\CSLRightInline}[1]{\parbox[t]{\linewidth - \csllabelwidth}{\strut#1\strut}}
\newcommand{\CSLIndent}[1]{\hspace{\cslhangindent}#1}

\makeatletter
\@ifpackageloaded{tcolorbox}{}{\usepackage[skins,breakable]{tcolorbox}}
\@ifpackageloaded{fontawesome5}{}{\usepackage{fontawesome5}}
\definecolor{quarto-callout-color}{HTML}{909090}
\definecolor{quarto-callout-note-color}{HTML}{0758E5}
\definecolor{quarto-callout-important-color}{HTML}{CC1914}
\definecolor{quarto-callout-warning-color}{HTML}{EB9113}
\definecolor{quarto-callout-tip-color}{HTML}{00A047}
\definecolor{quarto-callout-caution-color}{HTML}{FC5300}
\definecolor{quarto-callout-color-frame}{HTML}{acacac}
\definecolor{quarto-callout-note-color-frame}{HTML}{4582ec}
\definecolor{quarto-callout-important-color-frame}{HTML}{d9534f}
\definecolor{quarto-callout-warning-color-frame}{HTML}{f0ad4e}
\definecolor{quarto-callout-tip-color-frame}{HTML}{02b875}
\definecolor{quarto-callout-caution-color-frame}{HTML}{fd7e14}
\makeatother
\makeatletter
\@ifpackageloaded{caption}{}{\usepackage{caption}}
\AtBeginDocument{%
\ifdefined\contentsname
  \renewcommand*\contentsname{Table of contents}
\else
  \newcommand\contentsname{Table of contents}
\fi
\ifdefined\listfigurename
  \renewcommand*\listfigurename{List of Figures}
\else
  \newcommand\listfigurename{List of Figures}
\fi
\ifdefined\listtablename
  \renewcommand*\listtablename{List of Tables}
\else
  \newcommand\listtablename{List of Tables}
\fi
\ifdefined\figurename
  \renewcommand*\figurename{Figure}
\else
  \newcommand\figurename{Figure}
\fi
\ifdefined\tablename
  \renewcommand*\tablename{Table}
\else
  \newcommand\tablename{Table}
\fi
}
\@ifpackageloaded{float}{}{\usepackage{float}}
\floatstyle{ruled}
\@ifundefined{c@chapter}{\newfloat{codelisting}{h}{lop}}{\newfloat{codelisting}{h}{lop}[chapter]}
\floatname{codelisting}{Listing}
\newcommand*\listoflistings{\listof{codelisting}{List of Listings}}
\makeatother
\makeatletter
\makeatother
\makeatletter
\@ifpackageloaded{caption}{}{\usepackage{caption}}
\@ifpackageloaded{subcaption}{}{\usepackage{subcaption}}
\makeatother

\ifLuaTeX
  \usepackage{selnolig}  % disable illegal ligatures
\fi
\usepackage{bookmark}

\IfFileExists{xurl.sty}{\usepackage{xurl}}{} % add URL line breaks if available
\urlstyle{same} % disable monospaced font for URLs
\hypersetup{
  pdftitle={SDS},
  hidelinks,
  pdfcreator={LaTeX via pandoc}}


\title{SDS}
\usepackage{etoolbox}
\makeatletter
\providecommand{\subtitle}[1]{% add subtitle to \maketitle
  \apptocmd{\@title}{\par {\large #1 \par}}{}{}
}
\makeatother
\subtitle{Secondary data from centralized routine registry data
available on remote research server}
\author{}
\date{}

\begin{document}
\frontmatter
\maketitle

\renewcommand*\contentsname{Table of contents}
{
\setcounter{tocdepth}{2}
\tableofcontents
}

\mainmatter
\chapter{Sundhedsdatastyrelsen (SDS)/The Danish Health Data
Authority}\label{sundhedsdatastyrelsen-sdsthe-danish-health-data-authority}

Sundhedsdatastyrelsen - FSEID-00006014.

Download the ``udtræksbeskrivelse'' here:\\
Dataspecifikation \\
Dataoversigt

\chapter{Population}\label{population}

The DD2 data are uploaded to SDS, and they check the validity of the
CPR-numbers before encrypting them. The dataset
\texttt{DS\_EXT\_DD2\_POP} includes all individuals with a valid
CPR-number. Individuals with a non-valid CPR-number are still included
in the uploaded data, e.g.~in \texttt{DS\_EXT\_DD2CORE}, but they will
have a missing encrypted CPR-number (\texttt{CPR\_ENCRYPTED}) and can
thus not be linked across data sources. Previously (when we had local
data), we would notify DD2 (Odense) about non-valid CPR-numbers, but
this is no longer possible as we are not allowed to identify the
individuals based on individual-level data on the servers.

The data files uploaded by us have the prefix \texttt{DS\_EXT\_}. Data
are available in 2 versions; the initial upload version (without suffix)
and the updated versions (with suffix \_230601, \_231026, etc.,
corresponding to the date of the update on SDS). The registry data
(views) will always be based on the newest population, i.e., individuals
in the newest version of \texttt{DS\_EXT\_DD2\_POP}.

\chapter{Rawdata available on SDS}\label{rawdata-available-on-sds}

Data on SDS are based on data views, i.e., we have access to data
updated (almost) on a day-to-day basis. The update file (SAS
\texttt{IN06014.OpdateringsOversigt}) contains the time of the current
update for each dataset in the library. The metadata file (SAS
\texttt{metadata.metadata\_allcolumns}) contains information about
variables in the data. For research projects, in order to be able to
reproduce results, remember to save a copy of the research data in a
suitable project folder on SDS.

\begin{tcolorbox}[enhanced jigsaw, toprule=.15mm, titlerule=0mm, toptitle=1mm, arc=.35mm, coltitle=black, colbacktitle=quarto-callout-note-color!10!white, left=2mm, title=\textcolor{quarto-callout-note-color}{\faInfo}\hspace{0.5em}{Note}, colframe=quarto-callout-note-color-frame, rightrule=.15mm, bottomtitle=1mm, opacitybacktitle=0.6, bottomrule=.15mm, leftrule=.75mm, breakable, opacityback=0, colback=white]

Using SAS, if you extract data from views you will likely save a lot of
time by writing code that is proper sql, rather than using SAS-specific
syntax

\end{tcolorbox}

\begin{Shaded}
\begin{Highlighting}[]
\NormalTok{Example}\SpecialCharTok{:}
\DecValTok{1}\ErrorTok{)}

\NormalTok{proc sql;}
\NormalTok{  create table x as}
\NormalTok{  select }\SpecialCharTok{*}
\NormalTok{  from }\FunctionTok{data.lab\_lab\_dm\_forsker}\NormalTok{(}\AttributeTok{where =}\NormalTok{ (analysiscode }\ControlFlowTok{in}\SpecialCharTok{:}\NormalTok{ ([list of npucodes])));}
\NormalTok{quit; }\SpecialCharTok{*}\NormalTok{ takes }\SpecialCharTok{\textasciitilde{}}\DecValTok{40}\NormalTok{ minutes to run}

\DecValTok{2}\ErrorTok{)}
\NormalTok{proc sql;}
\NormalTok{  create table x as}
\NormalTok{  select }\SpecialCharTok{*}
\NormalTok{  from data.lab\_lab\_dm\_forsker}
\NormalTok{  where analysiscode eqt [npucode1] or ... or analysiscode eqt [npucodeN];}
\NormalTok{quit; }\SpecialCharTok{*}\NormalTok{ takes }\SpecialCharTok{\textasciitilde{}}\DecValTok{40}\NormalTok{ minutes to run}


\DecValTok{3}\ErrorTok{)}
\NormalTok{proc sql;}
\NormalTok{  create table x as}
\NormalTok{  select }\SpecialCharTok{*}
\NormalTok{  from data.lab\_lab\_dm\_forsker}
\NormalTok{  where analysiscode like [npucode1}\SpecialCharTok{\%] or ... or analysiscode like [npucodeN\%}\NormalTok{];}
\NormalTok{quit; }\SpecialCharTok{*}\NormalTok{ takes }\SpecialCharTok{\textasciitilde{}}\DecValTok{10{-}20}\NormalTok{ seconds to run}
\end{Highlighting}
\end{Shaded}

Further documentation on the main data sources can be found here:

\begin{itemize}
\item
  CPR-registret (CPR):
  \href{https://sundhedsdatastyrelsen.dk/da/registre-og-services/om-de-nationale-sundhedsregistre/personoplysninger-og-sundhedsfaglig-beskaeftigelse/cpr-registeret}{documentation}\\
  References: Schmidt et al. (2014)
\item
  Landspatientregistret (LPR2 and LPR3):
  \href{https://sundhedsdatastyrelsen.dk/da/registre-og-services/om-de-nationale-sundhedsregistre/sygdomme-laegemidler-og-behandlinger/landspatientregisteret}{documentation},
  \href{https://sundhedsdatastyrelsen.dk/da/forskerservice/om-forskerservice/nyheder_forskerservice/helbredsforloeb_lpr3f070422}{LPR3},
  \href{https://sundhedsdatastyrelsen.dk/-/media/sds/filer/Forskerservice/Vejledning\%20til\%20LPR3_F}{LPR3
  vejledning}\\
  References: Schmidt et al. (2015)
\item
  Lægemiddelstatistikregistret (LRS/LMS):
  \href{https://sundhedsdatastyrelsen.dk/da/registre-og-services/om-de-nationale-sundhedsregistre/sygdomme-laegemidler-og-behandlinger/laegemiddelstatistikregisteret}{documentation}\\
  References: Kildemoes et al. (2011), Johannesdottir et al. (2012),
  Pottegård et al. (2017), Ehrenstein et al. (2010)
\item
  Laboratoriedatabasen (LAB-F/RLRR):
  \href{https://sundhedsdatastyrelsen.dk/da/registre-og-services/om-de-nationale-sundhedsregistre/doedsaarsager-og-biologisk-materiale/laboratoriedatabasen}{documentation}\\
  References: Grann et al. (2011), Arendt et al. (2020)
\end{itemize}

\section{Access to data}\label{access-to-data}

All DD2-researchers with access to FSEID-00006014 can (in principle)
work with all the datasets. As these datasets are considered rawdata,
they are cannot be edited/changed/moved/renamed/etc. To access data,
please do the following:

\begin{itemize}
\item
  \textbf{SAS}: The data are already available in the libraries
  \texttt{in06014} and \texttt{metadata}
\item
  \textbf{STATA}: Run the (relevant parts of the) DO file located at
  \texttt{F:\textbackslash{}Projekter\textbackslash{}FSEID00006014\textbackslash{}DB\ Connections\ FSED00006014.do}
\item
  \textbf{R}: Run the (relevant parts of the) R file located at
  \texttt{F:\textbackslash{}Projekter\textbackslash{}FSEID00006014\textbackslash{}DB\ Connections\ FSED00006014.R}
\end{itemize}

In each research project, it is the individual researchers'
responsibility to ensure data usage is allowed.

\chapter{Data uploads}\label{data-uploads}

The first data were uploaded in 2022 when the server access was
initially established. Since then, the data have been updated multiple
times.

\begin{longtable}[]{@{}
  >{\raggedright\arraybackslash}p{(\columnwidth - 6\tabcolsep) * \real{0.2432}}
  >{\raggedright\arraybackslash}p{(\columnwidth - 6\tabcolsep) * \real{0.2703}}
  >{\raggedright\arraybackslash}p{(\columnwidth - 6\tabcolsep) * \real{0.2432}}
  >{\raggedleft\arraybackslash}p{(\columnwidth - 6\tabcolsep) * \real{0.2432}}@{}}
\toprule\noalign{}
\begin{minipage}[b]{\linewidth}\raggedright
Short description
\end{minipage} & \begin{minipage}[b]{\linewidth}\raggedright
Detailed description
\end{minipage} & \begin{minipage}[b]{\linewidth}\raggedright
Date
\end{minipage} & \begin{minipage}[b]{\linewidth}\raggedleft
Affected datasets
\end{minipage} \\
\midrule\noalign{}
\endhead
\bottomrule\noalign{}
\endlastfoot
Original upload & Upload of datasets for the initial server access &
Summer 2022 & DD2CORE, DD2\_POP, BIOBANK, DDDA, FOEDSELSDATA, IDNC,
IL6\_TNF \\
General update & Population update and inclusion of data source
FODSTATUS. The biobank was updated with additional measurements of CRP,
c-peptide, glucose, and HOMA (from April 2022), and additional
``pladebiomarkører'' were added to IL6\_TNF. DDDA data were updated
(June 2022) before the registry closed. IDNC was trimmed to the current
population. & June 2023 (\_230621) & DD2CORE, DD2\_POP, BIOBANK, DDDA,
FOEDSELSDATA, IDNC, IL6\_TNF, FODSTATUS \\
Upload of biobank data & C-peptide and glucose (and therefore also HOMA)
were changed, and the remaining ``pladebiomarkører'' were added to
IL6\_TNF, which is now complete & Fall 2023 (\_231026) & BIOBANK,
IL6\_TNF \\
Upload of biobank data (genetics) & Genetic variables (polygenic risk
scores) were added to IL6\_TNF (long format). All new variables are
labelled ``Gen1'' & April 2024 (\_240429) & IL6\_TNF \\
Upload of biobank data (genetics) and DICTA variable & Additional
genetic variables were added to IL6\_TNF. In DD2CORE the DICTA variable
was added. & July 2024 (\_240711) & DD2CORE, IL6\_TNF \\
Population update and upload of genetics & The population (and
registries) was updated (November 2023) and additional genetic variables
were included. We made the SDS update in the same round where the DST
data were updated (genindstilling via DDV), and decided to rename
IL6\_TNF to BIOMARK including only the ``pladebiomarkører'', and
including the genetic variables in a separate dataset called GENETIK.
The dataset INFODATA including e.g., DICTA was also included. & December
2024 (upload not available as we are waiting for SDS updates) & DD2CORE,
DD2\_POP, BIOMARK, GENETIK, INFODATA \\
December 2024 upload & The upload from December 2024 was made available
and we added CAR and DAR from the ``omlægning af registre'' (still
waiting for LAB). & June 2025 & DD2CORE, DD2\_POP, BIOMARK, GENETIK,
INFODATA (and CAR+DAR from registries) \\
\end{longtable}

NB: we try to distinguish between ``update'' and ``upload''. An
``update'' means that the population, and therefore also the registry
data, has changed, and that the entire DD2 cohort should be seen as a
new version. By ``upload'' we mean additional data uploaded to the
current population. This could for example be adding analyses from the
biobank or adding a new data source, i.e., something that doesn't affect
the population itself. The plan is to make one annual update
(November/end of the year) and two to three additional uploads during
per year.

\phantomsection\label{refs}
\begin{CSLReferences}{1}{1}
\bibitem[\citeproctext]{ref-Arendt_ClinEpi_2020}
Arendt JFH, Hansen AT, Ladefoged SA, Sørensen HT, Pedersen L, Adelborg
K. \href{https://doi.org/10.2147/clep.S245060}{Existing data sources in
clinical epidemiology: Laboratory information system databases in
denmark}. Clin Epidemiol. 2020;12:469--75.

\bibitem[\citeproctext]{ref-Ehrenstein_ClinEpi_2010}
Ehrenstein V, Antonsen S, Pedersen L.
\href{https://doi.org/10.2147/CLEP.S13458}{Existing data sources for
clinical epidemiology: Aarhus university prescription database}. Clin
Epidemiol. 2010;2:273--9.

\bibitem[\citeproctext]{ref-Grann_ClinEpi_2011}
Grann AF, Erichsen R, Nielsen AG, Froslev T, Thomsen RW.
\href{https://doi.org/10.2147/CLEP.S17901}{Existing data sources for
clinical epidemiology: The clinical laboratory information system
(LABKA) research database at aarhus university, denmark}. Clin
Epidemiol. 2011;3:133--8.

\bibitem[\citeproctext]{ref-Johannesdottir_ClinEpi_2012}
Johannesdottir SA, Horvath-Puho E, Ehrenstein V, Schmidt M, Pedersen L,
Sorensen HT. \href{https://doi.org/10.2147/CLEP.S37587}{Existing data
sources for clinical epidemiology: The danish national database of
reimbursed prescriptions}. Clin Epidemiol. 2012;4:303--13.

\bibitem[\citeproctext]{ref-Kildemoes_ScanJPubHeal_2011}
Kildemoes HW, Sørensen HT, Hallas J.
\href{https://doi.org/10.1177/1403494810394717}{The danish national
prescription registry}. Scand J Public Health. 2011;39(7 Suppl):38--41.

\bibitem[\citeproctext]{ref-Pottegard_IntJEpi_2017}
Pottegård A, Schmidt SAJ, Wallach-Kildemoes H, Sørensen HT, Hallas J,
Schmidt M. \href{https://doi.org/10.1093/ije/dyw213}{Data resource
profile: The danish national prescription registry}. Int J Epidemiol.
2017;46(3):798--798f.

\bibitem[\citeproctext]{ref-Schmidt_EurJEpi_2014}
Schmidt M, Pedersen L, Sørensen HT.
\href{https://doi.org/10.1007/s10654-014-9930-3}{The danish civil
registration system as a tool in epidemiology}. Eur J Epidemiol.
2014;29(8):541--9.

\bibitem[\citeproctext]{ref-Schmidt_ClinEpi_2015}
Schmidt M, Schmidt SA, Sandegaard JL, Ehrenstein V, Pedersen L, Sorensen
HT. \href{https://doi.org/10.2147/CLEP.S91125}{The danish national
patient registry: A review of content, data quality, and research
potential}. Clin Epidemiol. 2015;7:449--90.

\end{CSLReferences}


\backmatter


\end{document}
