% Options for packages loaded elsewhere
\PassOptionsToPackage{unicode}{hyperref}
\PassOptionsToPackage{hyphens}{url}
%
\documentclass[
  11pt,
  a4paper,
]{book}

\usepackage{amsmath,amssymb}
\usepackage{iftex}
\ifPDFTeX
  \usepackage[T1]{fontenc}
  \usepackage[utf8]{inputenc}
  \usepackage{textcomp} % provide euro and other symbols
\else % if luatex or xetex
  \usepackage{unicode-math}
  \defaultfontfeatures{Scale=MatchLowercase}
  \defaultfontfeatures[\rmfamily]{Ligatures=TeX,Scale=1}
\fi
\usepackage{lmodern}
\ifPDFTeX\else  
    % xetex/luatex font selection
\fi
% Use upquote if available, for straight quotes in verbatim environments
\IfFileExists{upquote.sty}{\usepackage{upquote}}{}
\IfFileExists{microtype.sty}{% use microtype if available
  \usepackage[]{microtype}
  \UseMicrotypeSet[protrusion]{basicmath} % disable protrusion for tt fonts
}{}
\makeatletter
\@ifundefined{KOMAClassName}{% if non-KOMA class
  \IfFileExists{parskip.sty}{%
    \usepackage{parskip}
  }{% else
    \setlength{\parindent}{0pt}
    \setlength{\parskip}{6pt plus 2pt minus 1pt}}
}{% if KOMA class
  \KOMAoptions{parskip=half}}
\makeatother
\usepackage{xcolor}
\setlength{\emergencystretch}{3em} % prevent overfull lines
\setcounter{secnumdepth}{5}
% Make \paragraph and \subparagraph free-standing
\makeatletter
\ifx\paragraph\undefined\else
  \let\oldparagraph\paragraph
  \renewcommand{\paragraph}{
    \@ifstar
      \xxxParagraphStar
      \xxxParagraphNoStar
  }
  \newcommand{\xxxParagraphStar}[1]{\oldparagraph*{#1}\mbox{}}
  \newcommand{\xxxParagraphNoStar}[1]{\oldparagraph{#1}\mbox{}}
\fi
\ifx\subparagraph\undefined\else
  \let\oldsubparagraph\subparagraph
  \renewcommand{\subparagraph}{
    \@ifstar
      \xxxSubParagraphStar
      \xxxSubParagraphNoStar
  }
  \newcommand{\xxxSubParagraphStar}[1]{\oldsubparagraph*{#1}\mbox{}}
  \newcommand{\xxxSubParagraphNoStar}[1]{\oldsubparagraph{#1}\mbox{}}
\fi
\makeatother


\providecommand{\tightlist}{%
  \setlength{\itemsep}{0pt}\setlength{\parskip}{0pt}}\usepackage{longtable,booktabs,array}
\usepackage{calc} % for calculating minipage widths
% Correct order of tables after \paragraph or \subparagraph
\usepackage{etoolbox}
\makeatletter
\patchcmd\longtable{\par}{\if@noskipsec\mbox{}\fi\par}{}{}
\makeatother
% Allow footnotes in longtable head/foot
\IfFileExists{footnotehyper.sty}{\usepackage{footnotehyper}}{\usepackage{footnote}}
\makesavenoteenv{longtable}
\usepackage{graphicx}
\makeatletter
\def\maxwidth{\ifdim\Gin@nat@width>\linewidth\linewidth\else\Gin@nat@width\fi}
\def\maxheight{\ifdim\Gin@nat@height>\textheight\textheight\else\Gin@nat@height\fi}
\makeatother
% Scale images if necessary, so that they will not overflow the page
% margins by default, and it is still possible to overwrite the defaults
% using explicit options in \includegraphics[width, height, ...]{}
\setkeys{Gin}{width=\maxwidth,height=\maxheight,keepaspectratio}
% Set default figure placement to htbp
\makeatletter
\def\fps@figure{htbp}
\makeatother
% definitions for citeproc citations
\NewDocumentCommand\citeproctext{}{}
\NewDocumentCommand\citeproc{mm}{%
  \begingroup\def\citeproctext{#2}\cite{#1}\endgroup}
\makeatletter
 % allow citations to break across lines
 \let\@cite@ofmt\@firstofone
 % avoid brackets around text for \cite:
 \def\@biblabel#1{}
 \def\@cite#1#2{{#1\if@tempswa , #2\fi}}
\makeatother
\newlength{\cslhangindent}
\setlength{\cslhangindent}{1.5em}
\newlength{\csllabelwidth}
\setlength{\csllabelwidth}{3em}
\newenvironment{CSLReferences}[2] % #1 hanging-indent, #2 entry-spacing
 {\begin{list}{}{%
  \setlength{\itemindent}{0pt}
  \setlength{\leftmargin}{0pt}
  \setlength{\parsep}{0pt}
  % turn on hanging indent if param 1 is 1
  \ifodd #1
   \setlength{\leftmargin}{\cslhangindent}
   \setlength{\itemindent}{-1\cslhangindent}
  \fi
  % set entry spacing
  \setlength{\itemsep}{#2\baselineskip}}}
 {\end{list}}
\usepackage{calc}
\newcommand{\CSLBlock}[1]{\hfill\break\parbox[t]{\linewidth}{\strut\ignorespaces#1\strut}}
\newcommand{\CSLLeftMargin}[1]{\parbox[t]{\csllabelwidth}{\strut#1\strut}}
\newcommand{\CSLRightInline}[1]{\parbox[t]{\linewidth - \csllabelwidth}{\strut#1\strut}}
\newcommand{\CSLIndent}[1]{\hspace{\cslhangindent}#1}

\makeatletter
\@ifpackageloaded{caption}{}{\usepackage{caption}}
\AtBeginDocument{%
\ifdefined\contentsname
  \renewcommand*\contentsname{Table of contents}
\else
  \newcommand\contentsname{Table of contents}
\fi
\ifdefined\listfigurename
  \renewcommand*\listfigurename{List of Figures}
\else
  \newcommand\listfigurename{List of Figures}
\fi
\ifdefined\listtablename
  \renewcommand*\listtablename{List of Tables}
\else
  \newcommand\listtablename{List of Tables}
\fi
\ifdefined\figurename
  \renewcommand*\figurename{Figure}
\else
  \newcommand\figurename{Figure}
\fi
\ifdefined\tablename
  \renewcommand*\tablename{Table}
\else
  \newcommand\tablename{Table}
\fi
}
\@ifpackageloaded{float}{}{\usepackage{float}}
\floatstyle{ruled}
\@ifundefined{c@chapter}{\newfloat{codelisting}{h}{lop}}{\newfloat{codelisting}{h}{lop}[chapter]}
\floatname{codelisting}{Listing}
\newcommand*\listoflistings{\listof{codelisting}{List of Listings}}
\makeatother
\makeatletter
\makeatother
\makeatletter
\@ifpackageloaded{caption}{}{\usepackage{caption}}
\@ifpackageloaded{subcaption}{}{\usepackage{subcaption}}
\makeatother
\makeatletter
\@ifpackageloaded{sidenotes}{}{\usepackage{sidenotes}}
\@ifpackageloaded{marginnote}{}{\usepackage{marginnote}}
\makeatother

\ifLuaTeX
  \usepackage{selnolig}  % disable illegal ligatures
\fi
\usepackage{bookmark}

\IfFileExists{xurl.sty}{\usepackage{xurl}}{} % add URL line breaks if available
\urlstyle{same} % disable monospaced font for URLs
\hypersetup{
  pdftitle={DD2 enrollment},
  hidelinks,
  pdfcreator={LaTeX via pandoc}}


\title{DD2 enrollment}
\author{}
\date{}

\begin{document}
\frontmatter
\maketitle

\renewcommand*\contentsname{Table of contents}
{
\setcounter{tocdepth}{2}
\tableofcontents
}

\mainmatter
Enrollment in DD2 includes a questionnaire and the collection of urine
and blood samples. The questionnaire includes a section with a few
questions for the patient to fill out (at registration for enrollment)
and another to be filled out by the enroller (at enrollment). All
individuals older than 18 years of age diagnosed with diabetes within
the last 2 years are eligible for inclusion in DD2 (previously, a longer
diabetes duration was allowed).

Individuals may either sign up online and be contacted later, or be
approached by practice personnel during routine care. The individual
will receive an e-mail with a link to WebPatient, where a few questions
should be filled out before the questionnaire will be sent to the
practice. Hereafter, a DD2 examination will take place, where additional
questions will be answered (e.g., a clinically assessed date of diabetes
onset) along with a physical examination (e.g., to measure weight, hip
and waist circumference, and resting heart rate). The examination also
involves a urine sample (morning urine) and a blood sample (fasting).

Enrollment can either take place in the general practice (GP) or in the
outpatient clinic. Registration is also possible for individuals without
internet access. For further information about the enrollment, please
see the DD2 website (Danish):
\href{https://dd2.dk/praksis-og-amb/patientinformation}{Patient
information} and
\href{https://dd2.dk/praksis-og-amb/vejledninger-og-reg-skema}{Instructions
and questionnaires}.

\begin{figure*}%
\begin{figure}[H]

{\centering \includegraphics[width=0.6\textwidth,height=\textheight]{../figures/DK_indrullering.png}

}

\caption{Figure from \url{https://dd2.dk/om-dd2/hvem-er-med} (downloaded
30 October 2025) showing the general practitioners and outpatient clinic
enrolling individuals in DD2.}

\end{figure}%%
\end{figure*}%

The first individual was enrolled in DD2 in November 2010. Between 2010
and April 2023, the DD2 cohort enrolled 11,381 individuals. An
additional 717 individuals were enrolled by November 2023, yielding a
total of N=12,098 individuals. In the 2024 cohort paper (Kristensen et
al. (2024)), we illustrated the enrollment during the period from
November 2010 to April 2023.

\begin{figure*}%
\begin{figure}[H]

{\centering \includegraphics[width=0.5\textwidth,height=\textheight]{../figures/enrollment_axis.png}

}

\caption{Figure from Kristensen et al. (2024) with a total of N=11,369
individuals. Overview of enrollment in the DD2 cohort from November 2010
to April 2023. The plots show the cumulative and annual number of
enrolled individuals.}

\end{figure}%%
\end{figure*}%

The median age at enrollment was 61 years (IQR 53--69 years) and 41\%
were women (Kristensen et al. (2024)). Individuals are eligible for
inclusion in DD2 if they are diagnosed with diabetes within the last 2
years (a longer diabetes duration was allowed in the early phase).
Median diabetes duration at time of enrollment was 1.3 years (IQR
0.4--2.9 years) and 87\% were already prescribed with glucose-lowering
treatment at the time of enrollment (Kristensen et al. (2024)).

For more information about the enrollment and data collection, please
contact DD2. Some of the general DD2 documents can also be downloaded
here (Danish, downloaded 12 January 2024):

\begin{itemize}
\tightlist
\item
  DD2 questionnaire:
\item
  Inclusion information:
\item
  Biobank samples: and
\end{itemize}

\phantomsection\label{refs}
\begin{CSLReferences}{1}{1}
\bibitem[\citeproctext]{ref-Kristensen_ClinEpi_2024}
Kristensen FPB, Nicolaisen SK, Nielsen JS, Christensen DH, Højlund K,
Beck-Nielsen H, et al. \href{https://doi.org/10.2147/CLEP.S469958}{The
danish centre for strategic research in type 2 diabetes (DD2) project
cohort and biobank from 2010 through 2023 - a cohort profile update}.
2024;16:641--56.

\end{CSLReferences}


\backmatter


\end{document}
