% Options for packages loaded elsewhere
\PassOptionsToPackage{unicode}{hyperref}
\PassOptionsToPackage{hyphens}{url}
%
\documentclass[
  11pt,
  a4paper,
]{book}

\usepackage{amsmath,amssymb}
\usepackage{iftex}
\ifPDFTeX
  \usepackage[T1]{fontenc}
  \usepackage[utf8]{inputenc}
  \usepackage{textcomp} % provide euro and other symbols
\else % if luatex or xetex
  \usepackage{unicode-math}
  \defaultfontfeatures{Scale=MatchLowercase}
  \defaultfontfeatures[\rmfamily]{Ligatures=TeX,Scale=1}
\fi
\usepackage{lmodern}
\ifPDFTeX\else  
    % xetex/luatex font selection
\fi
% Use upquote if available, for straight quotes in verbatim environments
\IfFileExists{upquote.sty}{\usepackage{upquote}}{}
\IfFileExists{microtype.sty}{% use microtype if available
  \usepackage[]{microtype}
  \UseMicrotypeSet[protrusion]{basicmath} % disable protrusion for tt fonts
}{}
\makeatletter
\@ifundefined{KOMAClassName}{% if non-KOMA class
  \IfFileExists{parskip.sty}{%
    \usepackage{parskip}
  }{% else
    \setlength{\parindent}{0pt}
    \setlength{\parskip}{6pt plus 2pt minus 1pt}}
}{% if KOMA class
  \KOMAoptions{parskip=half}}
\makeatother
\usepackage{xcolor}
\setlength{\emergencystretch}{3em} % prevent overfull lines
\setcounter{secnumdepth}{5}
% Make \paragraph and \subparagraph free-standing
\makeatletter
\ifx\paragraph\undefined\else
  \let\oldparagraph\paragraph
  \renewcommand{\paragraph}{
    \@ifstar
      \xxxParagraphStar
      \xxxParagraphNoStar
  }
  \newcommand{\xxxParagraphStar}[1]{\oldparagraph*{#1}\mbox{}}
  \newcommand{\xxxParagraphNoStar}[1]{\oldparagraph{#1}\mbox{}}
\fi
\ifx\subparagraph\undefined\else
  \let\oldsubparagraph\subparagraph
  \renewcommand{\subparagraph}{
    \@ifstar
      \xxxSubParagraphStar
      \xxxSubParagraphNoStar
  }
  \newcommand{\xxxSubParagraphStar}[1]{\oldsubparagraph*{#1}\mbox{}}
  \newcommand{\xxxSubParagraphNoStar}[1]{\oldsubparagraph{#1}\mbox{}}
\fi
\makeatother


\providecommand{\tightlist}{%
  \setlength{\itemsep}{0pt}\setlength{\parskip}{0pt}}\usepackage{longtable,booktabs,array}
\usepackage{calc} % for calculating minipage widths
% Correct order of tables after \paragraph or \subparagraph
\usepackage{etoolbox}
\makeatletter
\patchcmd\longtable{\par}{\if@noskipsec\mbox{}\fi\par}{}{}
\makeatother
% Allow footnotes in longtable head/foot
\IfFileExists{footnotehyper.sty}{\usepackage{footnotehyper}}{\usepackage{footnote}}
\makesavenoteenv{longtable}
\usepackage{graphicx}
\makeatletter
\def\maxwidth{\ifdim\Gin@nat@width>\linewidth\linewidth\else\Gin@nat@width\fi}
\def\maxheight{\ifdim\Gin@nat@height>\textheight\textheight\else\Gin@nat@height\fi}
\makeatother
% Scale images if necessary, so that they will not overflow the page
% margins by default, and it is still possible to overwrite the defaults
% using explicit options in \includegraphics[width, height, ...]{}
\setkeys{Gin}{width=\maxwidth,height=\maxheight,keepaspectratio}
% Set default figure placement to htbp
\makeatletter
\def\fps@figure{htbp}
\makeatother
% definitions for citeproc citations
\NewDocumentCommand\citeproctext{}{}
\NewDocumentCommand\citeproc{mm}{%
  \begingroup\def\citeproctext{#2}\cite{#1}\endgroup}
\makeatletter
 % allow citations to break across lines
 \let\@cite@ofmt\@firstofone
 % avoid brackets around text for \cite:
 \def\@biblabel#1{}
 \def\@cite#1#2{{#1\if@tempswa , #2\fi}}
\makeatother
\newlength{\cslhangindent}
\setlength{\cslhangindent}{1.5em}
\newlength{\csllabelwidth}
\setlength{\csllabelwidth}{3em}
\newenvironment{CSLReferences}[2] % #1 hanging-indent, #2 entry-spacing
 {\begin{list}{}{%
  \setlength{\itemindent}{0pt}
  \setlength{\leftmargin}{0pt}
  \setlength{\parsep}{0pt}
  % turn on hanging indent if param 1 is 1
  \ifodd #1
   \setlength{\leftmargin}{\cslhangindent}
   \setlength{\itemindent}{-1\cslhangindent}
  \fi
  % set entry spacing
  \setlength{\itemsep}{#2\baselineskip}}}
 {\end{list}}
\usepackage{calc}
\newcommand{\CSLBlock}[1]{\hfill\break\parbox[t]{\linewidth}{\strut\ignorespaces#1\strut}}
\newcommand{\CSLLeftMargin}[1]{\parbox[t]{\csllabelwidth}{\strut#1\strut}}
\newcommand{\CSLRightInline}[1]{\parbox[t]{\linewidth - \csllabelwidth}{\strut#1\strut}}
\newcommand{\CSLIndent}[1]{\hspace{\cslhangindent}#1}

\makeatletter
\@ifpackageloaded{caption}{}{\usepackage{caption}}
\AtBeginDocument{%
\ifdefined\contentsname
  \renewcommand*\contentsname{Table of contents}
\else
  \newcommand\contentsname{Table of contents}
\fi
\ifdefined\listfigurename
  \renewcommand*\listfigurename{List of Figures}
\else
  \newcommand\listfigurename{List of Figures}
\fi
\ifdefined\listtablename
  \renewcommand*\listtablename{List of Tables}
\else
  \newcommand\listtablename{List of Tables}
\fi
\ifdefined\figurename
  \renewcommand*\figurename{Figure}
\else
  \newcommand\figurename{Figure}
\fi
\ifdefined\tablename
  \renewcommand*\tablename{Table}
\else
  \newcommand\tablename{Table}
\fi
}
\@ifpackageloaded{float}{}{\usepackage{float}}
\floatstyle{ruled}
\@ifundefined{c@chapter}{\newfloat{codelisting}{h}{lop}}{\newfloat{codelisting}{h}{lop}[chapter]}
\floatname{codelisting}{Listing}
\newcommand*\listoflistings{\listof{codelisting}{List of Listings}}
\makeatother
\makeatletter
\makeatother
\makeatletter
\@ifpackageloaded{caption}{}{\usepackage{caption}}
\@ifpackageloaded{subcaption}{}{\usepackage{subcaption}}
\makeatother
\makeatletter
\@ifpackageloaded{sidenotes}{}{\usepackage{sidenotes}}
\@ifpackageloaded{marginnote}{}{\usepackage{marginnote}}
\makeatother

\ifLuaTeX
  \usepackage{selnolig}  % disable illegal ligatures
\fi
\usepackage{bookmark}

\IfFileExists{xurl.sty}{\usepackage{xurl}}{} % add URL line breaks if available
\urlstyle{same} % disable monospaced font for URLs
\hypersetup{
  pdftitle={Data structure},
  hidelinks,
  pdfcreator={LaTeX via pandoc}}


\title{Data structure}
\author{}
\date{}

\begin{document}
\frontmatter
\maketitle

\renewcommand*\contentsname{Table of contents}
{
\setcounter{tocdepth}{2}
\tableofcontents
}

\mainmatter
The full DD2 research data are based on data collected and linked from
multiple sources (Kristensen et al. (2024)). This page gives an overview
of the data structure, the data sources, and the definitions of data
types. Detailed descriptions of each data source are provided on the
following pages.

Within the DD2 data collection, the data sources include \textbf{primary
data}, i.e., data unique to DD2, including data and material collected
from participants at enrollment, and additional data from research
studies within the DD2 cohort. The data also include \textbf{secondary
data}, i.e., external data collected from other data sources and then
linked to DD2. Data from the DD2 data collection are also linked to the
nationwide routine registry data located in centralized databases on
remote research servers (SDS and DST).

\begin{figure*}%
\begin{figure}[H]

{\centering \includegraphics[width=0.9\textwidth,height=\textheight]{../figures/Figure_1.png}

}

\caption{Overview of data sources in the DD2 cohort (Kristensen et al.
(2024)). Notes: *These data can be accessed through collaboration with
DD2 research groups.}

\end{figure}%%
\end{figure*}%

\chapter{Primary data}\label{primary-data}

Primary data can be split into two categories:

\section{1. DD2 enrollment material}\label{dd2-enrollment-material}

Everything collected during the DD2 enrollment (questionnaire responses,
clinical measurements, and biological samples) constitutes the DD2
enrollment material and forms the foundation of DD2 as a distinct data
source. This includes:

\begin{itemize}
\item
  DD2 interview data (questionnaire)
\item
  Data from the clinical examination at enrollment
\item
  Blood and urine samples collected for storage in the DD2 biobank in
  Vejle.
\end{itemize}

Enrollment in DD2 involves completing the DD2 questionnaire, undergoing
the clinical examination, and providing blood and urine samples. The DD2
cohort is defined by individuals who completed the enrollment process.
These materials are unique to DD2 and are not available from other
sources. All blood and urine samples are stored in the DD2 biobank and
may be used for future research, though analysis typically depends on
funding from specific studies.

\section{2. Additional DD2 data}\label{additional-dd2-data}

Research studies in the DD2 setup can generate new primary data. The
completeness and availability of these primary data depend on when the
studies are conducted and when results become available. The data are
generated among individuals already enrolled in DD2, and are likely not
collected during the initial enrollment process. This could e.g., be
data from studies collecting additional follow-up data (by a follow-up
questionnaire or follow-up clinical examinations) or from intervention
studies including individuals from the DD2 cohort.\\
The biological samples (blood and urine) are collected at enrollment and
thus considered enrollment material. However, they are stored in the
biobank, and can be used for additional analyses in new research
studies, e.g., biomarker analyses, genetic studies, or DNA sequencing.
These analysis results add to the primary collected data by adding
additional baseline information.

Examples of currently available additional DD2 data include

\begin{itemize}
\item
  IDNC questionnaire data, a follow-up questionnaire from 2016 including
  (follow-up) questions from the initial questionnaire along with
  multiple questions concerning diabetic neuropathy.
\item
  IDA, DICTA, CsubT2D, etc., are examples of projects initiated within
  DD2. These projects generate their own additional data and analyses
  but data are still part of the overall DD2 setup.
\item
  QoL data. After enrollment, an e-mail with a questionnaire about
  quality of life (QoL) is sent to the individual. Additional follow-up
  questionnaires are sent after 2 years, 4 years, and 6 years.
\end{itemize}

These data sources will most likely not be routinely updated with new
individuals or additional data, as they are primarily generated as part
of specific research projects with closed cohorts. The data can be
accessed through collaboration with DD2 research groups.

\chapter{Secondary data}\label{secondary-data}

Secondary data refer to data from external but non-centralized data
sources. They were not initially generated with the intention to be
included in DD2. However, as they may be relevant to DD2 research
questions, data from these external data sources are linked to the DD2.
In practice, this means that data (or a subset of data) are transferred
from the external data source to DD2, providing information on the
individuals (or a subset of individuals) in DD2. Linkage to external
data sources is crucial in DD2, as it allows for information about
variables that are resource-intensive to collect in clinical practice
and unavailable in the Danish health registries. We currently have data
from the following external data sources:

\begin{itemize}
\item
  The Danish Diabetes Database for Adults (DDDA, closed by June 2022)
\item
  The Danish National Podiatrist Registry (fodstatusdatabasen)
\item
  The Danish National Archive (Rigsarkivet), currently including birth
  data from midwife journals.
\end{itemize}

In general, updates from these sources may be possible, for example by
extending follow-up periods or including additional individuals. The
secondary data can be accessed through collaboration with DD2 research
groups. Additional agreements may be established with Regionernes
Kliniske Kvalitetsudviklingsprogram (RKKP) to enable access to clinical
quality data, such as those from the Dansk Diabetes Database (DDiD) and
the Dansk kvalitetsdatabase for diabetisk retinopati (DiaBase). Access
depends on funding, data availability, and formal collaboration
agreements.

\chapter{Routine Registry Data}\label{routine-registry-data}

The routine registry data are also secondary data, however, they are
only available on remote research servers and not considered part of the
DD2 data collection. DD2 research data is uploaded to the servers hosted
at Danmarks Statistik (DST, Statistics Denmark) and
Sundhedsdatastyrelsen (SDS, The Danish Health Data Authority). On these
servers, we have access to Danish medical and administrative registries,
such as data from the CPR registry, hospital data from LPR, prescription
data, laboratory data etc. Linkage to additional registries and
databases is possible after approval of data permission applications
sent to Statistics Denmark or the Danish Health Data Authority.

\phantomsection\label{refs}
\begin{CSLReferences}{1}{1}
\bibitem[\citeproctext]{ref-Kristensen_ClinEpi_2024}
Kristensen FPB, Nicolaisen SK, Nielsen JS, Christensen DH, Højlund K,
Beck-Nielsen H, et al. \href{https://doi.org/10.2147/CLEP.S469958}{The
danish centre for strategic research in type 2 diabetes (DD2) project
cohort and biobank from 2010 through 2023 - a cohort profile update}.
2024;16:641--56.

\end{CSLReferences}


\backmatter


\end{document}
