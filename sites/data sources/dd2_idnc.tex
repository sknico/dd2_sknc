% Options for packages loaded elsewhere
\PassOptionsToPackage{unicode}{hyperref}
\PassOptionsToPackage{hyphens}{url}
%
\documentclass[
  11pt,
  a4paper,
]{book}

\usepackage{amsmath,amssymb}
\usepackage{iftex}
\ifPDFTeX
  \usepackage[T1]{fontenc}
  \usepackage[utf8]{inputenc}
  \usepackage{textcomp} % provide euro and other symbols
\else % if luatex or xetex
  \usepackage{unicode-math}
  \defaultfontfeatures{Scale=MatchLowercase}
  \defaultfontfeatures[\rmfamily]{Ligatures=TeX,Scale=1}
\fi
\usepackage{lmodern}
\ifPDFTeX\else  
    % xetex/luatex font selection
\fi
% Use upquote if available, for straight quotes in verbatim environments
\IfFileExists{upquote.sty}{\usepackage{upquote}}{}
\IfFileExists{microtype.sty}{% use microtype if available
  \usepackage[]{microtype}
  \UseMicrotypeSet[protrusion]{basicmath} % disable protrusion for tt fonts
}{}
\makeatletter
\@ifundefined{KOMAClassName}{% if non-KOMA class
  \IfFileExists{parskip.sty}{%
    \usepackage{parskip}
  }{% else
    \setlength{\parindent}{0pt}
    \setlength{\parskip}{6pt plus 2pt minus 1pt}}
}{% if KOMA class
  \KOMAoptions{parskip=half}}
\makeatother
\usepackage{xcolor}
\setlength{\emergencystretch}{3em} % prevent overfull lines
\setcounter{secnumdepth}{5}
% Make \paragraph and \subparagraph free-standing
\makeatletter
\ifx\paragraph\undefined\else
  \let\oldparagraph\paragraph
  \renewcommand{\paragraph}{
    \@ifstar
      \xxxParagraphStar
      \xxxParagraphNoStar
  }
  \newcommand{\xxxParagraphStar}[1]{\oldparagraph*{#1}\mbox{}}
  \newcommand{\xxxParagraphNoStar}[1]{\oldparagraph{#1}\mbox{}}
\fi
\ifx\subparagraph\undefined\else
  \let\oldsubparagraph\subparagraph
  \renewcommand{\subparagraph}{
    \@ifstar
      \xxxSubParagraphStar
      \xxxSubParagraphNoStar
  }
  \newcommand{\xxxSubParagraphStar}[1]{\oldsubparagraph*{#1}\mbox{}}
  \newcommand{\xxxSubParagraphNoStar}[1]{\oldsubparagraph{#1}\mbox{}}
\fi
\makeatother


\providecommand{\tightlist}{%
  \setlength{\itemsep}{0pt}\setlength{\parskip}{0pt}}\usepackage{longtable,booktabs,array}
\usepackage{calc} % for calculating minipage widths
% Correct order of tables after \paragraph or \subparagraph
\usepackage{etoolbox}
\makeatletter
\patchcmd\longtable{\par}{\if@noskipsec\mbox{}\fi\par}{}{}
\makeatother
% Allow footnotes in longtable head/foot
\IfFileExists{footnotehyper.sty}{\usepackage{footnotehyper}}{\usepackage{footnote}}
\makesavenoteenv{longtable}
\usepackage{graphicx}
\makeatletter
\def\maxwidth{\ifdim\Gin@nat@width>\linewidth\linewidth\else\Gin@nat@width\fi}
\def\maxheight{\ifdim\Gin@nat@height>\textheight\textheight\else\Gin@nat@height\fi}
\makeatother
% Scale images if necessary, so that they will not overflow the page
% margins by default, and it is still possible to overwrite the defaults
% using explicit options in \includegraphics[width, height, ...]{}
\setkeys{Gin}{width=\maxwidth,height=\maxheight,keepaspectratio}
% Set default figure placement to htbp
\makeatletter
\def\fps@figure{htbp}
\makeatother
% definitions for citeproc citations
\NewDocumentCommand\citeproctext{}{}
\NewDocumentCommand\citeproc{mm}{%
  \begingroup\def\citeproctext{#2}\cite{#1}\endgroup}
\makeatletter
 % allow citations to break across lines
 \let\@cite@ofmt\@firstofone
 % avoid brackets around text for \cite:
 \def\@biblabel#1{}
 \def\@cite#1#2{{#1\if@tempswa , #2\fi}}
\makeatother
\newlength{\cslhangindent}
\setlength{\cslhangindent}{1.5em}
\newlength{\csllabelwidth}
\setlength{\csllabelwidth}{3em}
\newenvironment{CSLReferences}[2] % #1 hanging-indent, #2 entry-spacing
 {\begin{list}{}{%
  \setlength{\itemindent}{0pt}
  \setlength{\leftmargin}{0pt}
  \setlength{\parsep}{0pt}
  % turn on hanging indent if param 1 is 1
  \ifodd #1
   \setlength{\leftmargin}{\cslhangindent}
   \setlength{\itemindent}{-1\cslhangindent}
  \fi
  % set entry spacing
  \setlength{\itemsep}{#2\baselineskip}}}
 {\end{list}}
\usepackage{calc}
\newcommand{\CSLBlock}[1]{\hfill\break\parbox[t]{\linewidth}{\strut\ignorespaces#1\strut}}
\newcommand{\CSLLeftMargin}[1]{\parbox[t]{\csllabelwidth}{\strut#1\strut}}
\newcommand{\CSLRightInline}[1]{\parbox[t]{\linewidth - \csllabelwidth}{\strut#1\strut}}
\newcommand{\CSLIndent}[1]{\hspace{\cslhangindent}#1}

\makeatletter
\@ifpackageloaded{caption}{}{\usepackage{caption}}
\AtBeginDocument{%
\ifdefined\contentsname
  \renewcommand*\contentsname{Table of contents}
\else
  \newcommand\contentsname{Table of contents}
\fi
\ifdefined\listfigurename
  \renewcommand*\listfigurename{List of Figures}
\else
  \newcommand\listfigurename{List of Figures}
\fi
\ifdefined\listtablename
  \renewcommand*\listtablename{List of Tables}
\else
  \newcommand\listtablename{List of Tables}
\fi
\ifdefined\figurename
  \renewcommand*\figurename{Figure}
\else
  \newcommand\figurename{Figure}
\fi
\ifdefined\tablename
  \renewcommand*\tablename{Table}
\else
  \newcommand\tablename{Table}
\fi
}
\@ifpackageloaded{float}{}{\usepackage{float}}
\floatstyle{ruled}
\@ifundefined{c@chapter}{\newfloat{codelisting}{h}{lop}}{\newfloat{codelisting}{h}{lop}[chapter]}
\floatname{codelisting}{Listing}
\newcommand*\listoflistings{\listof{codelisting}{List of Listings}}
\makeatother
\makeatletter
\makeatother
\makeatletter
\@ifpackageloaded{caption}{}{\usepackage{caption}}
\@ifpackageloaded{subcaption}{}{\usepackage{subcaption}}
\makeatother

\ifLuaTeX
  \usepackage{selnolig}  % disable illegal ligatures
\fi
\usepackage{bookmark}

\IfFileExists{xurl.sty}{\usepackage{xurl}}{} % add URL line breaks if available
\urlstyle{same} % disable monospaced font for URLs
\hypersetup{
  pdftitle={Neuropathy questionnaire (IDNC)},
  hidelinks,
  pdfcreator={LaTeX via pandoc}}


\title{Neuropathy questionnaire (IDNC)}
\usepackage{etoolbox}
\makeatletter
\providecommand{\subtitle}[1]{% add subtitle to \maketitle
  \apptocmd{\@title}{\par {\large #1 \par}}{}{}
}
\makeatother
\subtitle{Primary data derived from DD2 research studies}
\author{}
\date{}

\begin{document}
\frontmatter
\maketitle

\renewcommand*\contentsname{Table of contents}
{
\setcounter{tocdepth}{2}
\tableofcontents
}

\mainmatter
Go to \href{dd2_idnc.qmd\#data-documentation}{Data documentation}

IDNC is short for ``International Diabetic Neuropathy Consortium''. Read
more about it here: \url{https://idnc.au.dk/about-idnc}.

A list of 7,011 CPR-numbers from all individuals enrolled in DD2 by
February were sent to the CPR registry to get their addresses. The only
criteria to be included in the questionnaire study were that individuals
should be enrolled in DD2 (February 2016), be alive (24 May 2016), and
have a valid Danish address (24 May 2016). A total of 6,726
questionnaires were sent out in June 2016 (07 June 2016), and a reminder
was sent in September 2016 (12 September 2016) and again in October 2016
(10 October 2016) to those who had not provided a response. A total of
5,755 questionnaires were returned during the fall of 2016 and the
inclusion ended in January 2017 (24 January 2017). The questionnaire
included questions about height, weight, smoking and alcohol, and
physical activity. These questions were intended to be follow-up
questions similar to those from the baseline questionnaire. In addition,
the questionnaire included questions about falls, quality of life,
sleep, mental health, neuropathy (in hand and feet) and pain. The
questionnaire thus included the Michigan Neuropathy Screening Instrument
questionnaire (MNSIq) and the Douleur Neuropathique en 4 Questions
(DN4).

Read more about the questionnaire in Gylfadottir et al. (2020) and
Christensen et al. (2020).\\
The questionnaire (in Danish) can be downloaded here: \\
An English translation of the questionnaire is included in the
supplement to Gylfadottir et al. (2020).

The questions and variable definitions for the original variables can be
downloaded here: \\

\begin{center}\rule{0.5\linewidth}{0.5pt}\end{center}

\chapter{Data documentation}\label{data-documentation}

\section{IDNC.sas7bdat}\label{idnc.sas7bdat}

\begin{longtable}[]{@{}
  >{\raggedright\arraybackslash}p{(\columnwidth - 6\tabcolsep) * \real{0.3151}}
  >{\raggedright\arraybackslash}p{(\columnwidth - 6\tabcolsep) * \real{0.2192}}
  >{\raggedright\arraybackslash}p{(\columnwidth - 6\tabcolsep) * \real{0.2329}}
  >{\raggedright\arraybackslash}p{(\columnwidth - 6\tabcolsep) * \real{0.2329}}@{}}
\toprule\noalign{}
\begin{minipage}[b]{\linewidth}\raggedright
Format (var x obs)
\end{minipage} & \begin{minipage}[b]{\linewidth}\raggedright
Id variables
\end{minipage} & \begin{minipage}[b]{\linewidth}\raggedright
Unique key
\end{minipage} & \begin{minipage}[b]{\linewidth}\raggedright
Important dates
\end{minipage} \\
\midrule\noalign{}
\endhead
\bottomrule\noalign{}
\endlastfoot
Wide (162 x 5,658) & CPR & CPR & reg\_dato, IDNCData\_date \\
\end{longtable}

The \texttt{IDNCData\_date} takes the value \texttt{01JUN2016} for all
individuals, as this is the date of the first IDNC data round. The local
data also include the variable \texttt{tid} with the values \texttt{T1},
\texttt{T2} and \texttt{T3}, indicating in which round the individual
answered the questionnaire (based on whether or not a reminder was
sent). This variable is not included in the datasets on the servers.

\begin{longtable}[]{@{}lllll@{}}
\caption{Illustration of the overall data structure. The dataset is in
wide format (162 x 5,658), with CPR as the unique key.}\tabularnewline
\toprule\noalign{}
Row & CPR & IDNCData\_date & reg\_dato & \ldots{} \\
\midrule\noalign{}
\endfirsthead
\toprule\noalign{}
Row & CPR & IDNCData\_date & reg\_dato & \ldots{} \\
\midrule\noalign{}
\endhead
\bottomrule\noalign{}
\endlastfoot
1 & CPR1 & 01JUN2016 & reg\_dato1 & \ldots{} \\
2 & CPR2 & 01JUN2016 & reg\_dato2 & \ldots{} \\
3 & CPR3 & 01JUN2016 & reg\_dato3 & \ldots{} \\
\ldots{} & \ldots{} & \ldots{} & \ldots{} & \ldots{} \\
12,098 & CPR12098 & 01JUN2016 & reg\_dato12098 & \ldots{} \\
\end{longtable}

The original data are stored locally at DCE and a slightly modified
version (IDNC.sas7bdat) has been uploaded to DST and SDS. The changes
are mainly due to similar definitions of variables, variable names and
labels, and to comply with the SDS/DST regulations.

\subsection{Variables}\label{variables}

The original questionnaire data included a total of around 109 variables
(including some variables from the data cleaning process added by the
working group). To ease working with data and to have some pre-defined
variables/definitions, an additional 78 variables were created by the
working group.

The SAS syntax for the definition of the additional variables is
available here: \\

And the corresponding SAS formats for the additional variables are
available here: \\

(SAS syntax files can be opened using notepad if SAS is not installed)

\phantomsection\label{refs}
\begin{CSLReferences}{1}{1}
\bibitem[\citeproctext]{ref-Christensen_DiabCare_2020}
Christensen DH, Knudsen ST, Gylfadottir SS, Christensen LB, Nielsen JS,
Beck-Nielsen H, et al.
\href{https://doi.org/10.2337/dc19-2277}{Metabolic factors, lifestyle
habits, and possible polyneuropathy in early type 2 diabetes: A
nationwide study of 5,249 patients in the danish centre for strategic
research in type 2 diabetes (DD2) cohort}. Diabetes Care.
2020;43(6):1266--75.

\bibitem[\citeproctext]{ref-Gylfadottir_pain_2020}
Gylfadottir SS, Christensen DH, Nicolaisen SK, Andersen H, Callaghan BC,
Itani M, et al.
\href{https://doi.org/10.1097/j.pain.0000000000001744}{Diabetic
polyneuropathy and pain, prevalence, and patient characteristics: A
cross-sectional questionnaire study of 5,514 patients with recently
diagnosed type 2 diabetes}. Pain. 2020;161(3):574--83.

\end{CSLReferences}


\backmatter


\end{document}
