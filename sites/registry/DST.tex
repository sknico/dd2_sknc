% Options for packages loaded elsewhere
\PassOptionsToPackage{unicode}{hyperref}
\PassOptionsToPackage{hyphens}{url}
%
\documentclass[
  11pt,
  a4paper,
]{book}

\usepackage{amsmath,amssymb}
\usepackage{iftex}
\ifPDFTeX
  \usepackage[T1]{fontenc}
  \usepackage[utf8]{inputenc}
  \usepackage{textcomp} % provide euro and other symbols
\else % if luatex or xetex
  \usepackage{unicode-math}
  \defaultfontfeatures{Scale=MatchLowercase}
  \defaultfontfeatures[\rmfamily]{Ligatures=TeX,Scale=1}
\fi
\usepackage{lmodern}
\ifPDFTeX\else  
    % xetex/luatex font selection
\fi
% Use upquote if available, for straight quotes in verbatim environments
\IfFileExists{upquote.sty}{\usepackage{upquote}}{}
\IfFileExists{microtype.sty}{% use microtype if available
  \usepackage[]{microtype}
  \UseMicrotypeSet[protrusion]{basicmath} % disable protrusion for tt fonts
}{}
\makeatletter
\@ifundefined{KOMAClassName}{% if non-KOMA class
  \IfFileExists{parskip.sty}{%
    \usepackage{parskip}
  }{% else
    \setlength{\parindent}{0pt}
    \setlength{\parskip}{6pt plus 2pt minus 1pt}}
}{% if KOMA class
  \KOMAoptions{parskip=half}}
\makeatother
\usepackage{xcolor}
\setlength{\emergencystretch}{3em} % prevent overfull lines
\setcounter{secnumdepth}{5}
% Make \paragraph and \subparagraph free-standing
\makeatletter
\ifx\paragraph\undefined\else
  \let\oldparagraph\paragraph
  \renewcommand{\paragraph}{
    \@ifstar
      \xxxParagraphStar
      \xxxParagraphNoStar
  }
  \newcommand{\xxxParagraphStar}[1]{\oldparagraph*{#1}\mbox{}}
  \newcommand{\xxxParagraphNoStar}[1]{\oldparagraph{#1}\mbox{}}
\fi
\ifx\subparagraph\undefined\else
  \let\oldsubparagraph\subparagraph
  \renewcommand{\subparagraph}{
    \@ifstar
      \xxxSubParagraphStar
      \xxxSubParagraphNoStar
  }
  \newcommand{\xxxSubParagraphStar}[1]{\oldsubparagraph*{#1}\mbox{}}
  \newcommand{\xxxSubParagraphNoStar}[1]{\oldsubparagraph{#1}\mbox{}}
\fi
\makeatother

\usepackage{color}
\usepackage{fancyvrb}
\newcommand{\VerbBar}{|}
\newcommand{\VERB}{\Verb[commandchars=\\\{\}]}
\DefineVerbatimEnvironment{Highlighting}{Verbatim}{commandchars=\\\{\}}
% Add ',fontsize=\small' for more characters per line
\usepackage{framed}
\definecolor{shadecolor}{RGB}{241,243,245}
\newenvironment{Shaded}{\begin{snugshade}}{\end{snugshade}}
\newcommand{\AlertTok}[1]{\textcolor[rgb]{0.68,0.00,0.00}{#1}}
\newcommand{\AnnotationTok}[1]{\textcolor[rgb]{0.37,0.37,0.37}{#1}}
\newcommand{\AttributeTok}[1]{\textcolor[rgb]{0.40,0.45,0.13}{#1}}
\newcommand{\BaseNTok}[1]{\textcolor[rgb]{0.68,0.00,0.00}{#1}}
\newcommand{\BuiltInTok}[1]{\textcolor[rgb]{0.00,0.23,0.31}{#1}}
\newcommand{\CharTok}[1]{\textcolor[rgb]{0.13,0.47,0.30}{#1}}
\newcommand{\CommentTok}[1]{\textcolor[rgb]{0.37,0.37,0.37}{#1}}
\newcommand{\CommentVarTok}[1]{\textcolor[rgb]{0.37,0.37,0.37}{\textit{#1}}}
\newcommand{\ConstantTok}[1]{\textcolor[rgb]{0.56,0.35,0.01}{#1}}
\newcommand{\ControlFlowTok}[1]{\textcolor[rgb]{0.00,0.23,0.31}{\textbf{#1}}}
\newcommand{\DataTypeTok}[1]{\textcolor[rgb]{0.68,0.00,0.00}{#1}}
\newcommand{\DecValTok}[1]{\textcolor[rgb]{0.68,0.00,0.00}{#1}}
\newcommand{\DocumentationTok}[1]{\textcolor[rgb]{0.37,0.37,0.37}{\textit{#1}}}
\newcommand{\ErrorTok}[1]{\textcolor[rgb]{0.68,0.00,0.00}{#1}}
\newcommand{\ExtensionTok}[1]{\textcolor[rgb]{0.00,0.23,0.31}{#1}}
\newcommand{\FloatTok}[1]{\textcolor[rgb]{0.68,0.00,0.00}{#1}}
\newcommand{\FunctionTok}[1]{\textcolor[rgb]{0.28,0.35,0.67}{#1}}
\newcommand{\ImportTok}[1]{\textcolor[rgb]{0.00,0.46,0.62}{#1}}
\newcommand{\InformationTok}[1]{\textcolor[rgb]{0.37,0.37,0.37}{#1}}
\newcommand{\KeywordTok}[1]{\textcolor[rgb]{0.00,0.23,0.31}{\textbf{#1}}}
\newcommand{\NormalTok}[1]{\textcolor[rgb]{0.00,0.23,0.31}{#1}}
\newcommand{\OperatorTok}[1]{\textcolor[rgb]{0.37,0.37,0.37}{#1}}
\newcommand{\OtherTok}[1]{\textcolor[rgb]{0.00,0.23,0.31}{#1}}
\newcommand{\PreprocessorTok}[1]{\textcolor[rgb]{0.68,0.00,0.00}{#1}}
\newcommand{\RegionMarkerTok}[1]{\textcolor[rgb]{0.00,0.23,0.31}{#1}}
\newcommand{\SpecialCharTok}[1]{\textcolor[rgb]{0.37,0.37,0.37}{#1}}
\newcommand{\SpecialStringTok}[1]{\textcolor[rgb]{0.13,0.47,0.30}{#1}}
\newcommand{\StringTok}[1]{\textcolor[rgb]{0.13,0.47,0.30}{#1}}
\newcommand{\VariableTok}[1]{\textcolor[rgb]{0.07,0.07,0.07}{#1}}
\newcommand{\VerbatimStringTok}[1]{\textcolor[rgb]{0.13,0.47,0.30}{#1}}
\newcommand{\WarningTok}[1]{\textcolor[rgb]{0.37,0.37,0.37}{\textit{#1}}}

\providecommand{\tightlist}{%
  \setlength{\itemsep}{0pt}\setlength{\parskip}{0pt}}\usepackage{longtable,booktabs,array}
\usepackage{calc} % for calculating minipage widths
% Correct order of tables after \paragraph or \subparagraph
\usepackage{etoolbox}
\makeatletter
\patchcmd\longtable{\par}{\if@noskipsec\mbox{}\fi\par}{}{}
\makeatother
% Allow footnotes in longtable head/foot
\IfFileExists{footnotehyper.sty}{\usepackage{footnotehyper}}{\usepackage{footnote}}
\makesavenoteenv{longtable}
\usepackage{graphicx}
\makeatletter
\def\maxwidth{\ifdim\Gin@nat@width>\linewidth\linewidth\else\Gin@nat@width\fi}
\def\maxheight{\ifdim\Gin@nat@height>\textheight\textheight\else\Gin@nat@height\fi}
\makeatother
% Scale images if necessary, so that they will not overflow the page
% margins by default, and it is still possible to overwrite the defaults
% using explicit options in \includegraphics[width, height, ...]{}
\setkeys{Gin}{width=\maxwidth,height=\maxheight,keepaspectratio}
% Set default figure placement to htbp
\makeatletter
\def\fps@figure{htbp}
\makeatother

\makeatletter
\@ifpackageloaded{tcolorbox}{}{\usepackage[skins,breakable]{tcolorbox}}
\@ifpackageloaded{fontawesome5}{}{\usepackage{fontawesome5}}
\definecolor{quarto-callout-color}{HTML}{909090}
\definecolor{quarto-callout-note-color}{HTML}{0758E5}
\definecolor{quarto-callout-important-color}{HTML}{CC1914}
\definecolor{quarto-callout-warning-color}{HTML}{EB9113}
\definecolor{quarto-callout-tip-color}{HTML}{00A047}
\definecolor{quarto-callout-caution-color}{HTML}{FC5300}
\definecolor{quarto-callout-color-frame}{HTML}{acacac}
\definecolor{quarto-callout-note-color-frame}{HTML}{4582ec}
\definecolor{quarto-callout-important-color-frame}{HTML}{d9534f}
\definecolor{quarto-callout-warning-color-frame}{HTML}{f0ad4e}
\definecolor{quarto-callout-tip-color-frame}{HTML}{02b875}
\definecolor{quarto-callout-caution-color-frame}{HTML}{fd7e14}
\makeatother
\makeatletter
\@ifpackageloaded{caption}{}{\usepackage{caption}}
\AtBeginDocument{%
\ifdefined\contentsname
  \renewcommand*\contentsname{Table of contents}
\else
  \newcommand\contentsname{Table of contents}
\fi
\ifdefined\listfigurename
  \renewcommand*\listfigurename{List of Figures}
\else
  \newcommand\listfigurename{List of Figures}
\fi
\ifdefined\listtablename
  \renewcommand*\listtablename{List of Tables}
\else
  \newcommand\listtablename{List of Tables}
\fi
\ifdefined\figurename
  \renewcommand*\figurename{Figure}
\else
  \newcommand\figurename{Figure}
\fi
\ifdefined\tablename
  \renewcommand*\tablename{Table}
\else
  \newcommand\tablename{Table}
\fi
}
\@ifpackageloaded{float}{}{\usepackage{float}}
\floatstyle{ruled}
\@ifundefined{c@chapter}{\newfloat{codelisting}{h}{lop}}{\newfloat{codelisting}{h}{lop}[chapter]}
\floatname{codelisting}{Listing}
\newcommand*\listoflistings{\listof{codelisting}{List of Listings}}
\makeatother
\makeatletter
\makeatother
\makeatletter
\@ifpackageloaded{caption}{}{\usepackage{caption}}
\@ifpackageloaded{subcaption}{}{\usepackage{subcaption}}
\makeatother

\ifLuaTeX
  \usepackage{selnolig}  % disable illegal ligatures
\fi
\usepackage{bookmark}

\IfFileExists{xurl.sty}{\usepackage{xurl}}{} % add URL line breaks if available
\urlstyle{same} % disable monospaced font for URLs
\hypersetup{
  pdftitle={DST},
  hidelinks,
  pdfcreator={LaTeX via pandoc}}


\title{DST}
\usepackage{etoolbox}
\makeatletter
\providecommand{\subtitle}[1]{% add subtitle to \maketitle
  \apptocmd{\@title}{\par {\large #1 \par}}{}{}
}
\makeatother
\subtitle{Secondary data from centralized routine registry data
available on remote research server}
\author{}
\date{}

\begin{document}
\frontmatter
\maketitle

\renewcommand*\contentsname{Table of contents}
{
\setcounter{tocdepth}{2}
\tableofcontents
}

\mainmatter
\chapter{Danmarks Statistik (DST)/Statistics
Denmark}\label{danmarks-statistik-dststatistics-denmark}

DST Projektnr. 708637 (and FSEID-00006159 transfer from SDS). The
project is placed under the ``Projektdatabaseordning (705004)''. The DD2
projects itself had the project number 708637 (via 705004), and the
agreement with DST makes registry data available. Data from Det
Psykiatriske Centrale Forskningsregister (PCR), Laboratoriedatabasens
Forskertabel (LAB-F), Overvågningsdata - COVID-19-testsvar fra SSI
(OVD\_SSI), Det Danske Vaccinationsregister - COVID-19-vaccinationsdata
fra SSI (DDV\_SSI), and Cancerregisteret (CAR) are tranferred from SDS
(FSEID-00006159). Data from
Lægemiddelstatistikregisteret/Lægemiddeldatabasen (LSR/LMDB) is included
via DDV (DST), but based on the agreement with SDS (FSEID-00006159).

\begin{tcolorbox}[enhanced jigsaw, arc=.35mm, title=\textcolor{quarto-callout-important-color}{\faExclamation}\hspace{0.5em}{Important}, colbacktitle=quarto-callout-important-color!10!white, opacitybacktitle=0.6, toprule=.15mm, opacityback=0, titlerule=0mm, bottomtitle=1mm, toptitle=1mm, colframe=quarto-callout-important-color-frame, breakable, rightrule=.15mm, bottomrule=.15mm, left=2mm, colback=white, leftrule=.75mm, coltitle=black]

Tjek om aftale + bilag må ligge offentlig tilgængeligt.

\end{tcolorbox}

Download the general agreement including the agreement about medicinal
data here:\\
Aftale (mail) \\
Bilag LMDB

\chapter{Population}\label{population}

The DD2 data are uploaded to DST, and the population include all
individuals in DD2CORE (encrypted CPR). Registry data (Raw data
-\textgreater{} Grunddata) is linked, and based on the individuals in
the dataset \texttt{pniveau}.

\section{Formats}\label{formats}

On DST formats from DST and IDNC should be included manually. Here is an
example of how it can be done in SAS

\begin{Shaded}
\begin{Highlighting}[]
\NormalTok{libname fmt }\StringTok{"E:\textbackslash{}Formater\textbackslash{}SAS formater i DAnmarks Statistik\textbackslash{}FORMATKATALOG"}\NormalTok{ access}\OtherTok{=}\NormalTok{readonly;}
\NormalTok{options fmtsearch}\OtherTok{=}\NormalTok{(fmt.disced fmt.times\_personstatistik fmt.dream fmt.statistikbank);}

\NormalTok{libname idnc }\StringTok{"path"}\NormalTok{ access}\OtherTok{=}\NormalTok{readonly;}
\NormalTok{options fmtsearch}\OtherTok{=}\NormalTok{(idnc.idnc\_formats);}
\end{Highlighting}
\end{Shaded}

\chapter{Data uploads}\label{data-uploads}

The first data were uploaded in 2022 when the server access was
initially established. Since then, DST has changed their system to
Danmarks Datavindue (DDV). In November 2024, we had to make a
``genindtilling'' via DDV and in principle make a new data agreement. We
used this opportunity to make some changes in the data structure (e.g.,
from IL6\_TNF to BIOMARK and GENETIK)

\begin{longtable}[]{@{}
  >{\raggedright\arraybackslash}p{(\columnwidth - 6\tabcolsep) * \real{0.2500}}
  >{\raggedright\arraybackslash}p{(\columnwidth - 6\tabcolsep) * \real{0.2500}}
  >{\raggedright\arraybackslash}p{(\columnwidth - 6\tabcolsep) * \real{0.2500}}
  >{\raggedleft\arraybackslash}p{(\columnwidth - 6\tabcolsep) * \real{0.2500}}@{}}
\toprule\noalign{}
\begin{minipage}[b]{\linewidth}\raggedright
Short description
\end{minipage} & \begin{minipage}[b]{\linewidth}\raggedright
Detailed description
\end{minipage} & \begin{minipage}[b]{\linewidth}\raggedright
Date
\end{minipage} & \begin{minipage}[b]{\linewidth}\raggedleft
Affected datasets
\end{minipage} \\
\midrule\noalign{}
\endhead
\bottomrule\noalign{}
\endlastfoot
Original upload & Upload of datasets for the initial server access. Data
from the original upload are located in the folder with ``Eksterne
data'' under the rawdata folder (Rawdata \textgreater{} 708637
\textgreater{} Eksterne data). DD2 data are in the 20230104\_FraSDS
folder, whereas SDS data are in the folder 20221020\_FraSDS. Registry
data are in the Grunddata folder & Summer 2022 & DD2CORE, DD2\_POP,
BIOBANK, DDDA, FOEDSELSDATA, IDNC, IL6\_TNF \\
General update (DDV) & Population update (November 2023), updated
biobank data (variables from April 2022), including the new data source
FODSTATUS, and a general update of registry data. IL6\_TNF was updated
with all 22 ``pladebiomarkører'' and renamed to BIOMARK. The dataset
GENETIK was also included. The dataset INFODATA including e.g., DICTA
was also included. & November 2024/March 2025 (upload available in the
folders 20250402\_FraSDS and 03032025). We are still waiting for
registry updates (mainly due to changes at SDS) & DD2CORE, BIOBANK,
DDDA, FOEDSELSDATA, IDNC, FODSTATUS, BIOMARK, GENETIK, INFODATA. \\
\end{longtable}

NB: we try to distinguish between ``update'' and ``upload''. An
``update'' means that the population, and therefore also the registry
data, has changed, and that the entire DD2 cohort should be seen as a
new version. By ``upload'' we mean additional data uploaded to the
current population. This could for example be adding analyses from the
biobank or adding a new data source, i.e., something that doesn't affect
the population itself. The plan is to make one annual update
(November/end of the year) and two to three additional uploads during
per year.

Registry data on the forskermaskine is updated regularly. LPR-data for
the previous calendar year (e.g.~2023) is available during
October/November (2024), i.e., if we prepare for a new population update
at the end of the year (2024), we will probably be able to get LPR-data
for the full previous calendar year (2023), when we receive data in the
beginning of the next year (2025). Other registries are updated at
different time points (e.g., LMDB is updated quarterly).


\backmatter


\end{document}
