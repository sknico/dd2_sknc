% Options for packages loaded elsewhere
\PassOptionsToPackage{unicode}{hyperref}
\PassOptionsToPackage{hyphens}{url}
%
\documentclass[
  11pt,
  a4paper,
]{book}

\usepackage{amsmath,amssymb}
\usepackage{iftex}
\ifPDFTeX
  \usepackage[T1]{fontenc}
  \usepackage[utf8]{inputenc}
  \usepackage{textcomp} % provide euro and other symbols
\else % if luatex or xetex
  \usepackage{unicode-math}
  \defaultfontfeatures{Scale=MatchLowercase}
  \defaultfontfeatures[\rmfamily]{Ligatures=TeX,Scale=1}
\fi
\usepackage{lmodern}
\ifPDFTeX\else  
    % xetex/luatex font selection
\fi
% Use upquote if available, for straight quotes in verbatim environments
\IfFileExists{upquote.sty}{\usepackage{upquote}}{}
\IfFileExists{microtype.sty}{% use microtype if available
  \usepackage[]{microtype}
  \UseMicrotypeSet[protrusion]{basicmath} % disable protrusion for tt fonts
}{}
\makeatletter
\@ifundefined{KOMAClassName}{% if non-KOMA class
  \IfFileExists{parskip.sty}{%
    \usepackage{parskip}
  }{% else
    \setlength{\parindent}{0pt}
    \setlength{\parskip}{6pt plus 2pt minus 1pt}}
}{% if KOMA class
  \KOMAoptions{parskip=half}}
\makeatother
\usepackage{xcolor}
\setlength{\emergencystretch}{3em} % prevent overfull lines
\setcounter{secnumdepth}{5}
% Make \paragraph and \subparagraph free-standing
\makeatletter
\ifx\paragraph\undefined\else
  \let\oldparagraph\paragraph
  \renewcommand{\paragraph}{
    \@ifstar
      \xxxParagraphStar
      \xxxParagraphNoStar
  }
  \newcommand{\xxxParagraphStar}[1]{\oldparagraph*{#1}\mbox{}}
  \newcommand{\xxxParagraphNoStar}[1]{\oldparagraph{#1}\mbox{}}
\fi
\ifx\subparagraph\undefined\else
  \let\oldsubparagraph\subparagraph
  \renewcommand{\subparagraph}{
    \@ifstar
      \xxxSubParagraphStar
      \xxxSubParagraphNoStar
  }
  \newcommand{\xxxSubParagraphStar}[1]{\oldsubparagraph*{#1}\mbox{}}
  \newcommand{\xxxSubParagraphNoStar}[1]{\oldsubparagraph{#1}\mbox{}}
\fi
\makeatother


\providecommand{\tightlist}{%
  \setlength{\itemsep}{0pt}\setlength{\parskip}{0pt}}\usepackage{longtable,booktabs,array}
\usepackage{calc} % for calculating minipage widths
% Correct order of tables after \paragraph or \subparagraph
\usepackage{etoolbox}
\makeatletter
\patchcmd\longtable{\par}{\if@noskipsec\mbox{}\fi\par}{}{}
\makeatother
% Allow footnotes in longtable head/foot
\IfFileExists{footnotehyper.sty}{\usepackage{footnotehyper}}{\usepackage{footnote}}
\makesavenoteenv{longtable}
\usepackage{graphicx}
\makeatletter
\def\maxwidth{\ifdim\Gin@nat@width>\linewidth\linewidth\else\Gin@nat@width\fi}
\def\maxheight{\ifdim\Gin@nat@height>\textheight\textheight\else\Gin@nat@height\fi}
\makeatother
% Scale images if necessary, so that they will not overflow the page
% margins by default, and it is still possible to overwrite the defaults
% using explicit options in \includegraphics[width, height, ...]{}
\setkeys{Gin}{width=\maxwidth,height=\maxheight,keepaspectratio}
% Set default figure placement to htbp
\makeatletter
\def\fps@figure{htbp}
\makeatother
% definitions for citeproc citations
\NewDocumentCommand\citeproctext{}{}
\NewDocumentCommand\citeproc{mm}{%
  \begingroup\def\citeproctext{#2}\cite{#1}\endgroup}
\makeatletter
 % allow citations to break across lines
 \let\@cite@ofmt\@firstofone
 % avoid brackets around text for \cite:
 \def\@biblabel#1{}
 \def\@cite#1#2{{#1\if@tempswa , #2\fi}}
\makeatother
\newlength{\cslhangindent}
\setlength{\cslhangindent}{1.5em}
\newlength{\csllabelwidth}
\setlength{\csllabelwidth}{3em}
\newenvironment{CSLReferences}[2] % #1 hanging-indent, #2 entry-spacing
 {\begin{list}{}{%
  \setlength{\itemindent}{0pt}
  \setlength{\leftmargin}{0pt}
  \setlength{\parsep}{0pt}
  % turn on hanging indent if param 1 is 1
  \ifodd #1
   \setlength{\leftmargin}{\cslhangindent}
   \setlength{\itemindent}{-1\cslhangindent}
  \fi
  % set entry spacing
  \setlength{\itemsep}{#2\baselineskip}}}
 {\end{list}}
\usepackage{calc}
\newcommand{\CSLBlock}[1]{\hfill\break\parbox[t]{\linewidth}{\strut\ignorespaces#1\strut}}
\newcommand{\CSLLeftMargin}[1]{\parbox[t]{\csllabelwidth}{\strut#1\strut}}
\newcommand{\CSLRightInline}[1]{\parbox[t]{\linewidth - \csllabelwidth}{\strut#1\strut}}
\newcommand{\CSLIndent}[1]{\hspace{\cslhangindent}#1}

\makeatletter
\@ifpackageloaded{caption}{}{\usepackage{caption}}
\AtBeginDocument{%
\ifdefined\contentsname
  \renewcommand*\contentsname{Table of contents}
\else
  \newcommand\contentsname{Table of contents}
\fi
\ifdefined\listfigurename
  \renewcommand*\listfigurename{List of Figures}
\else
  \newcommand\listfigurename{List of Figures}
\fi
\ifdefined\listtablename
  \renewcommand*\listtablename{List of Tables}
\else
  \newcommand\listtablename{List of Tables}
\fi
\ifdefined\figurename
  \renewcommand*\figurename{Figure}
\else
  \newcommand\figurename{Figure}
\fi
\ifdefined\tablename
  \renewcommand*\tablename{Table}
\else
  \newcommand\tablename{Table}
\fi
}
\@ifpackageloaded{float}{}{\usepackage{float}}
\floatstyle{ruled}
\@ifundefined{c@chapter}{\newfloat{codelisting}{h}{lop}}{\newfloat{codelisting}{h}{lop}[chapter]}
\floatname{codelisting}{Listing}
\newcommand*\listoflistings{\listof{codelisting}{List of Listings}}
\makeatother
\makeatletter
\makeatother
\makeatletter
\@ifpackageloaded{caption}{}{\usepackage{caption}}
\@ifpackageloaded{subcaption}{}{\usepackage{subcaption}}
\makeatother

\ifLuaTeX
  \usepackage{selnolig}  % disable illegal ligatures
\fi
\usepackage{bookmark}

\IfFileExists{xurl.sty}{\usepackage{xurl}}{} % add URL line breaks if available
\urlstyle{same} % disable monospaced font for URLs
\hypersetup{
  pdftitle={DDDA},
  hidelinks,
  pdfcreator={LaTeX via pandoc}}


\title{DDDA}
\usepackage{etoolbox}
\makeatletter
\providecommand{\subtitle}[1]{% add subtitle to \maketitle
  \apptocmd{\@title}{\par {\large #1 \par}}{}{}
}
\makeatother
\subtitle{Secondary data from non-centralized data source}
\author{}
\date{}

\begin{document}
\frontmatter
\maketitle

\renewcommand*\contentsname{Table of contents}
{
\setcounter{tocdepth}{2}
\tableofcontents
}

\mainmatter
Go to \href{ddda.qmd\#data-documentation}{Data documentation}

\textbf{Danish Diabetes Database for Adults} (DDDA) (also known as Dansk
Voksen Diabetes Database (DVDD) or Det Nationale Indikatorprojekt (NIP)
Diabetes) is a no longer active database from regionernes kliniske
kvalitetsudviklingsprogram (RKKP). It was closed by 30 June 2022. It has
now been replaced by Dansk Diabetes Database (DDD/DDiD), which has been
active since 1 July 2022. DDiD currently mainly uses registry data,
however, as additional data are still being recorded in the clinic, it
is worth checking the database in the future, as additional variables
might be included (e.g.~smoking, blood pressure, BMI, etc.).

Extracts from the
\href{https://www.rkkp-dokumentation.dk/Public/Databases.aspx?db=43&db2=1000000549}{online}
DDDA documentation (downloaded 07 September 2023) about the population
and variables can be downloaded here:

And additional documents:

See also Jørgensen et al. (2016) for further information about the
database.

\chapter{Data}\label{data}

At DCE, we have data from DDDA in the period from May 2005 to January
2022 based on the n=10,241 individuals enrolled in DD2 per January 2022.
As the database is now closed, this is the final data set and we will
not get any updates of it.

In the latest version of the DDDA data, around 80\% (n=8,512) of the
individuals in DD2 had at least one record in DDDA. The records in DDDA
are uniquely defined based on the CPR-number and the variable
\texttt{status\_dato} which is the date that defines the data entry.
There is a median of 3 (IQR 2-5) records per individual, 85\% had at
least one record prior to enrollment in DD2, and there was a median of
-56 days (IQR -413-294) from the enrollment date to the nearest
\texttt{status\_dato} (negative value meaning that \texttt{status\_dato}
was before enrollment in DD2). The \texttt{status\_dato} is not
necessarily the date of the examination/results/visit. Many of the
variables thus have an associated date variable that is different from
\texttt{status\_dato}.

When using data from DDDA, it is important to keep in mind the missing
data. Data may be missing either because it was not reported or because
the individual is simply not registered in DDDA. Thus, when reporting
data from DDDA, e.g., in a baseline table, remember to consider what
``100\%'' should represent, as the interpretation of the result varies;
is it x\% of the entire study population or x\% of the study population
also registered in DDDA.

In addition, the selection and timing of data should be considered.
Individuals registered in DDDA fulfill the criteria for this database
(see population documentation above) which might induce some selection
bias. Individuals in DDDA might not be as newly diagnosed as individuals
in DD2. The records in DDDA are defined based on \texttt{status\_dato},
which is ``den dato, hvor der årligt, i relation til databasen, gøres
status over hvornår personer senest har fået foretaget forskellige
relevante undersøgelser'' (see
\href{https://www.db-dokumentation-sundk.dk/Public/PopulationsSpecifikation21.aspx?db2=1000000549}{website}).
The dates listed for each of the examinations might be long after (or
before) enrollment in DD2, and for each scientific study question, it is
thus crucial to consider in which time period data can be considered
valid. An example could be to only include measurements in the period
from 5 years prior to enrollment to 6 months after enrollment and then
among these, include the one closest to the enrollment date as a
``baseline'' value. This depends on the specific study.

\begin{center}\rule{0.5\linewidth}{0.5pt}\end{center}

\chapter{Data documentation}\label{data-documentation}

\section{DDDA.sas7bdat}\label{ddda.sas7bdat}

\begin{longtable}[]{@{}
  >{\raggedright\arraybackslash}p{(\columnwidth - 6\tabcolsep) * \real{0.2361}}
  >{\raggedright\arraybackslash}p{(\columnwidth - 6\tabcolsep) * \real{0.2917}}
  >{\raggedright\arraybackslash}p{(\columnwidth - 6\tabcolsep) * \real{0.2361}}
  >{\raggedright\arraybackslash}p{(\columnwidth - 6\tabcolsep) * \real{0.2361}}@{}}
\toprule\noalign{}
\begin{minipage}[b]{\linewidth}\raggedright
Format (var x obs)
\end{minipage} & \begin{minipage}[b]{\linewidth}\raggedright
Id variables
\end{minipage} & \begin{minipage}[b]{\linewidth}\raggedright
Unique key
\end{minipage} & \begin{minipage}[b]{\linewidth}\raggedright
Important dates
\end{minipage} \\
\midrule\noalign{}
\endhead
\bottomrule\noalign{}
\endlastfoot
Wide (75 x 30,806) & CPR & CPR*status\_dato & status\_dato, specific
date variables \\
\end{longtable}

N=8,512 distinct CPR numbers with non-missing data are included in the
data, i.e., individuals with data in DDDA. A total of N=10,219
individuals are included (merged with DD2core). A record is unique based
on CPR and status\_dato, however, for many of the variables, it is worth
also considering the corresponding var\_dato.

\begin{longtable}[]{@{}
  >{\raggedright\arraybackslash}p{(\columnwidth - 10\tabcolsep) * \real{0.1667}}
  >{\raggedright\arraybackslash}p{(\columnwidth - 10\tabcolsep) * \real{0.1667}}
  >{\raggedright\arraybackslash}p{(\columnwidth - 10\tabcolsep) * \real{0.1667}}
  >{\raggedright\arraybackslash}p{(\columnwidth - 10\tabcolsep) * \real{0.1667}}
  >{\raggedright\arraybackslash}p{(\columnwidth - 10\tabcolsep) * \real{0.1667}}
  >{\raggedright\arraybackslash}p{(\columnwidth - 10\tabcolsep) * \real{0.1667}}@{}}
\caption{Illustration of the overall data structure. The dataset is in
wide format (75 variables × 30,806 rows), with CPR*status\_dato as the
unique key.}\tabularnewline
\toprule\noalign{}
\begin{minipage}[b]{\linewidth}\raggedright
Row
\end{minipage} & \begin{minipage}[b]{\linewidth}\raggedright
CPR
\end{minipage} & \begin{minipage}[b]{\linewidth}\raggedright
status\_dato
\end{minipage} & \begin{minipage}[b]{\linewidth}\raggedright
Var
\end{minipage} & \begin{minipage}[b]{\linewidth}\raggedright
Var\_dato
\end{minipage} & \begin{minipage}[b]{\linewidth}\raggedright
\ldots{}
\end{minipage} \\
\midrule\noalign{}
\endfirsthead
\toprule\noalign{}
\begin{minipage}[b]{\linewidth}\raggedright
Row
\end{minipage} & \begin{minipage}[b]{\linewidth}\raggedright
CPR
\end{minipage} & \begin{minipage}[b]{\linewidth}\raggedright
status\_dato
\end{minipage} & \begin{minipage}[b]{\linewidth}\raggedright
Var
\end{minipage} & \begin{minipage}[b]{\linewidth}\raggedright
Var\_dato
\end{minipage} & \begin{minipage}[b]{\linewidth}\raggedright
\ldots{}
\end{minipage} \\
\midrule\noalign{}
\endhead
\bottomrule\noalign{}
\endlastfoot
1 & CPR1 & status\_dato1.1 & num. & Var\_dato1.1 & \ldots{} \\
2 & CPR1 & status\_dato1.2 & num. & Var\_dato1.2 & \ldots{} \\
3 & CPR1 & status\_dato1.3 & num. & Var\_dato1.2 & \ldots{} \\
\ldots{} & \ldots{} & \ldots{} & \ldots{} & \ldots{} & \ldots{} \\
10 & CPR1 & status\_dato1.10 & (missing, no record) & (missing, no
record) & \ldots{} \\
11 & CPR2 & status\_dato2.1 & num. & Var\_dato2.1 & \ldots{} \\
12 & CPR2 & status\_dato2.2 & num. & Var\_dato2.2 & \ldots{} \\
\ldots{} & \ldots{} & \ldots{} & \ldots{} & \ldots{} & \ldots{} \\
30,806 & CPR10219 & (missing, not in DDDA) & (missing, not in DDDA) &
(missing, not in DDDA) & \ldots{} \\
\end{longtable}

\subsection{Variables from previous
versions}\label{variables-from-previous-versions}

In the current DDDA data, some variables from the database were not
delivered. It was decided that the best suitable solution was to include
these variables from older versions of data (and thus missing for some
new individuals).

\begin{itemize}
\item
  From the 2021 data: \texttt{shared\_care} and
  \texttt{Plasmakreatinin\_operator}
\item
  From the 2018 data (also missing in the 2021 delivery):
  \texttt{diag\_dato}, \texttt{HbA1c\_kode}, \texttt{HDLcholesterol},
  and \texttt{HDLcholesterol\_undersoegelse\_kod}. These variables are
  included with a ``\_2018'' postfix.
\end{itemize}

\phantomsection\label{refs}
\begin{CSLReferences}{1}{1}
\bibitem[\citeproctext]{ref-Jorgensen_ClinEpi_2016}
Jørgensen ME, Kristensen JK, Reventlov Husted G, Cerqueira C, Rossing P.
\href{https://doi.org/10.2147/clep.S99518}{The danish adult diabetes
registry}. Clin Epidemiol. 2016;8:429--34.

\end{CSLReferences}


\backmatter


\end{document}
