% Options for packages loaded elsewhere
\PassOptionsToPackage{unicode}{hyperref}
\PassOptionsToPackage{hyphens}{url}
%
\documentclass[
  11pt,
  a4paper,
]{book}

\usepackage{amsmath,amssymb}
\usepackage{iftex}
\ifPDFTeX
  \usepackage[T1]{fontenc}
  \usepackage[utf8]{inputenc}
  \usepackage{textcomp} % provide euro and other symbols
\else % if luatex or xetex
  \usepackage{unicode-math}
  \defaultfontfeatures{Scale=MatchLowercase}
  \defaultfontfeatures[\rmfamily]{Ligatures=TeX,Scale=1}
\fi
\usepackage{lmodern}
\ifPDFTeX\else  
    % xetex/luatex font selection
\fi
% Use upquote if available, for straight quotes in verbatim environments
\IfFileExists{upquote.sty}{\usepackage{upquote}}{}
\IfFileExists{microtype.sty}{% use microtype if available
  \usepackage[]{microtype}
  \UseMicrotypeSet[protrusion]{basicmath} % disable protrusion for tt fonts
}{}
\makeatletter
\@ifundefined{KOMAClassName}{% if non-KOMA class
  \IfFileExists{parskip.sty}{%
    \usepackage{parskip}
  }{% else
    \setlength{\parindent}{0pt}
    \setlength{\parskip}{6pt plus 2pt minus 1pt}}
}{% if KOMA class
  \KOMAoptions{parskip=half}}
\makeatother
\usepackage{xcolor}
\setlength{\emergencystretch}{3em} % prevent overfull lines
\setcounter{secnumdepth}{5}
% Make \paragraph and \subparagraph free-standing
\makeatletter
\ifx\paragraph\undefined\else
  \let\oldparagraph\paragraph
  \renewcommand{\paragraph}{
    \@ifstar
      \xxxParagraphStar
      \xxxParagraphNoStar
  }
  \newcommand{\xxxParagraphStar}[1]{\oldparagraph*{#1}\mbox{}}
  \newcommand{\xxxParagraphNoStar}[1]{\oldparagraph{#1}\mbox{}}
\fi
\ifx\subparagraph\undefined\else
  \let\oldsubparagraph\subparagraph
  \renewcommand{\subparagraph}{
    \@ifstar
      \xxxSubParagraphStar
      \xxxSubParagraphNoStar
  }
  \newcommand{\xxxSubParagraphStar}[1]{\oldsubparagraph*{#1}\mbox{}}
  \newcommand{\xxxSubParagraphNoStar}[1]{\oldsubparagraph{#1}\mbox{}}
\fi
\makeatother


\providecommand{\tightlist}{%
  \setlength{\itemsep}{0pt}\setlength{\parskip}{0pt}}\usepackage{longtable,booktabs,array}
\usepackage{calc} % for calculating minipage widths
% Correct order of tables after \paragraph or \subparagraph
\usepackage{etoolbox}
\makeatletter
\patchcmd\longtable{\par}{\if@noskipsec\mbox{}\fi\par}{}{}
\makeatother
% Allow footnotes in longtable head/foot
\IfFileExists{footnotehyper.sty}{\usepackage{footnotehyper}}{\usepackage{footnote}}
\makesavenoteenv{longtable}
\usepackage{graphicx}
\makeatletter
\def\maxwidth{\ifdim\Gin@nat@width>\linewidth\linewidth\else\Gin@nat@width\fi}
\def\maxheight{\ifdim\Gin@nat@height>\textheight\textheight\else\Gin@nat@height\fi}
\makeatother
% Scale images if necessary, so that they will not overflow the page
% margins by default, and it is still possible to overwrite the defaults
% using explicit options in \includegraphics[width, height, ...]{}
\setkeys{Gin}{width=\maxwidth,height=\maxheight,keepaspectratio}
% Set default figure placement to htbp
\makeatletter
\def\fps@figure{htbp}
\makeatother

\makeatletter
\@ifpackageloaded{caption}{}{\usepackage{caption}}
\AtBeginDocument{%
\ifdefined\contentsname
  \renewcommand*\contentsname{Table of contents}
\else
  \newcommand\contentsname{Table of contents}
\fi
\ifdefined\listfigurename
  \renewcommand*\listfigurename{List of Figures}
\else
  \newcommand\listfigurename{List of Figures}
\fi
\ifdefined\listtablename
  \renewcommand*\listtablename{List of Tables}
\else
  \newcommand\listtablename{List of Tables}
\fi
\ifdefined\figurename
  \renewcommand*\figurename{Figure}
\else
  \newcommand\figurename{Figure}
\fi
\ifdefined\tablename
  \renewcommand*\tablename{Table}
\else
  \newcommand\tablename{Table}
\fi
}
\@ifpackageloaded{float}{}{\usepackage{float}}
\floatstyle{ruled}
\@ifundefined{c@chapter}{\newfloat{codelisting}{h}{lop}}{\newfloat{codelisting}{h}{lop}[chapter]}
\floatname{codelisting}{Listing}
\newcommand*\listoflistings{\listof{codelisting}{List of Listings}}
\makeatother
\makeatletter
\makeatother
\makeatletter
\@ifpackageloaded{caption}{}{\usepackage{caption}}
\@ifpackageloaded{subcaption}{}{\usepackage{subcaption}}
\makeatother

\ifLuaTeX
  \usepackage{selnolig}  % disable illegal ligatures
\fi
\usepackage{bookmark}

\IfFileExists{xurl.sty}{\usepackage{xurl}}{} % add URL line breaks if available
\urlstyle{same} % disable monospaced font for URLs
\hypersetup{
  pdftitle={DD2 questionnaire},
  hidelinks,
  pdfcreator={LaTeX via pandoc}}


\title{DD2 questionnaire}
\usepackage{etoolbox}
\makeatletter
\providecommand{\subtitle}[1]{% add subtitle to \maketitle
  \apptocmd{\@title}{\par {\large #1 \par}}{}{}
}
\makeatother
\subtitle{Primary DD2 enrollment data}
\author{}
\date{}

\begin{document}
\frontmatter
\maketitle

\renewcommand*\contentsname{Table of contents}
{
\setcounter{tocdepth}{2}
\tableofcontents
}

\mainmatter
Go to \href{dd2_questionnaire.qmd\#data-documentation}{Data
documentation}

Data from the DD2 questionnaire include the responses from the
enrollment, information from the clinical examination along with general
information about the individual. The DD2 questionnaire can be
downloaded here:

As of summer 2025, DD2 was migrated to a new server setup at Steno
Diabetes Center Odense. DCE retains data from the previous setup.
Historically, data updates were received regularly, but from the end of
2025, DCE must submit a formal application to obtain new data.

\begin{center}\rule{0.5\linewidth}{0.5pt}\end{center}

\chapter{Data documentation}\label{data-documentation}

\section{DD2core.sas7bdat}\label{dd2core.sas7bdat}

\begin{longtable}[]{@{}
  >{\raggedright\arraybackslash}p{(\columnwidth - 6\tabcolsep) * \real{0.2740}}
  >{\raggedright\arraybackslash}p{(\columnwidth - 6\tabcolsep) * \real{0.3288}}
  >{\raggedright\arraybackslash}p{(\columnwidth - 6\tabcolsep) * \real{0.1644}}
  >{\raggedright\arraybackslash}p{(\columnwidth - 6\tabcolsep) * \real{0.2329}}@{}}
\toprule\noalign{}
\begin{minipage}[b]{\linewidth}\raggedright
Format (var x obs)
\end{minipage} & \begin{minipage}[b]{\linewidth}\raggedright
Id variables
\end{minipage} & \begin{minipage}[b]{\linewidth}\raggedright
Unique key
\end{minipage} & \begin{minipage}[b]{\linewidth}\raggedright
Important dates
\end{minipage} \\
\midrule\noalign{}
\endhead
\bottomrule\noalign{}
\endlastfoot
Wide (52 x 12,098) & CPR, ProjektID, ydernr & CPR & reg\_dato \\
\end{longtable}

Data from the DD2 enrollment questionnaire and clinical examination are
stored in the \texttt{dd2core.sas7bdat} dataset. (Note: The name
``dd2core'' is not ideal, as there is no definition of what is actually
considered ``core'' data in DD2. Therefore, the data should be referred
to as ``enrollment data''. The dataset cannot easily be renamed as it is
in its current form in the agreements with DST and SDS.)

Data (November 2023) include enrollment information on all N=12,098
individuals enrolled in DD2 by November 2023. Data are on wide format,
currently with 52 variables, and include one row per CPR number. A
ProjektID is also included for individuals with a blood/urine sample in
the biobank. The variable \texttt{ydernr} is also included as a
identifier and is therefore encrypted by SDS and DST. Individuals are
enrolled in DD2 on the enrollment date, \texttt{reg\_dato}. Data can be
updated (new procedure after 2025), and if so, more enrolled individuals
are included (same variable set). We have usually received a full new
dataset with the current DD2 population - this is to ensure we only have
data on individuals we are allowed to have data on, and to ensure any
manual corrections in the DD2 setup have been implemented in our data.
Data were received as Excel or CSV files and underwent initial data
cleaning before being saved in \texttt{dd2core.sas7bdat}.

\begin{longtable}[]{@{}lllll@{}}
\caption{Illustration of the overall data structure. The dataset is in
wide format (52 variables × 12,098 rows), with CPR as the unique
key.}\tabularnewline
\toprule\noalign{}
Row & CPR & ProjektID & reg\_dato & \ldots{} \\
\midrule\noalign{}
\endfirsthead
\toprule\noalign{}
Row & CPR & ProjektID & reg\_dato & \ldots{} \\
\midrule\noalign{}
\endhead
\bottomrule\noalign{}
\endlastfoot
1 & CPR1 & ProjektID1 & reg\_dato1 & \ldots{} \\
2 & CPR2 & ProjektID2 & reg\_dato2 & \ldots{} \\
3 & CPR3 & ProjektID3 & reg\_dato3 & \ldots{} \\
\ldots{} & \ldots{} & \ldots{} & \ldots{} & \ldots{} \\
12,098 & CPR12098 & ProjektID12098 & reg\_dato12098 & \ldots{} \\
\end{longtable}


\backmatter


\end{document}
